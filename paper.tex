% TEMPLATE for Usenix papers, specifically to meet requirements of
%  USENIX '05
% originally a template for producing IEEE-format articles using LaTeX.
%   written by Matthew Ward, CS Department, Worcester Polytechnic Institute.
% adapted by David Beazley for his excellent SWIG paper in Proceedings,
%   Tcl 96
% turned into a smartass generic template by De Clarke, with thanks to
%   both the above pioneers
% use at your own risk.  Complaints to /dev/null.
% make it two column with no page numbering, default is 10 point

% Munged by Fred Douglis <douglis@research.att.com> 10/97 to separate
% the .sty file from the LaTeX source template, so that people can
% more easily include the .sty file into an existing document.  Also
% changed to more closely follow the style guidelines as represented
% by the Word sample file. 

% Note that since 2010, USENIX does not require endnotes. If you want
% foot of page notes, don't include the endnotes package in the 
% usepackage command, below.

% This version uses the latex2e styles, not the very ancient 2.09 stuff.
\documentclass[letterpaper,twocolumn,10pt]{article}

\usepackage{usenix,epsfig,endnotes}

%\usepackage[utf8]{inputenc}
%\usepackage[T1]{fontenc}
%\usepackage{microtype}
%\usepackage{flushend}

%\usepackage{amsmath}
%\usepackage{amssymb}
\usepackage{pifont}

\usepackage{hyperref}
\usepackage{multirow}
\usepackage{listings}
\usepackage{booktabs}
%\usepackage{tikz}
\usepackage{xspace}

\usepackage{balance}

\newcommand{\psharp}{P\#\xspace}
\newcommand{\csharp}{C\#\xspace}

\newcommand{\cmark}{\ding{51}}
\newcommand{\xmark}{\ding{55}}

\usepackage{color}
\definecolor{light-gray}{gray}{0.40}
\newcommand{\SQComment}[1]{\textcolor{red}{Shaz: #1}}
\newcommand{\PDComment}[1]{\textcolor{blue}{Pantazis: #1}}
\newcommand{\MMComment}[1]{\textcolor{cyan}{Matt: #1}}
\newcommand{\SCComment}[1]{\textcolor{magenta}{Shuo: #1}}
\newcommand{\CHComment}[1]{\textcolor{blue}{Cheng: #1}}
\newcommand{\PTComment}[1]{\textcolor{light-gray}{Paul: #1}}

\definecolor{bluekeywords}{rgb}{0.13,0.13,1}
\definecolor{greencomments}{rgb}{0,0.5,0}
\definecolor{redstrings}{rgb}{0.9,0,0}

\lstset{language=[Sharp]C,
showspaces=false,
showtabs=false,
breaklines=true,
showstringspaces=false,
breakatwhitespace=true,
escapeinside={(*@}{@*)},
morekeywords={OnEvent, nameof, var, state, cold, hot},
commentstyle=\color{greencomments},
keywordstyle=\color{bluekeywords}\bfseries,
stringstyle=\color{redstrings},
basicstyle=\ttfamily\scriptsize,
columns=fullflexible
}

\begin{document}

%don't want date printed
\date{}

%make title bold and 14 pt font (Latex default is non-bold, 16 pt)
\title{\Large \bf Uncovering Distributed System Bugs during Testing (not Production!)}

%for single author (just remove % characters)
\author{
{\rm Pantazis Deligiannis$^\dagger$, Matt McCutchen$^\diamond$, Paul Thomson$^\dagger$, Shuo Chen$^\star$}\\
{\rm Alastair F. Donaldson$^\dagger$, John Erickson$^\star$, Cheng Huang$^\star$, Akash Lal$^\star$}\\
{\rm Rashmi Mudduluru$^\star$, Shaz Qadeer$^\star$, Wolfram Schulte$^\star$}\\\\
$^\dagger$Imperial College London, $^\diamond$Massachusetts Institute of Technology, $^\star$Microsoft\\
} % end author

\maketitle

% Use the following at camera-ready time to suppress page numbers.
% Comment it out when you first submit the paper for review.
\thispagestyle{empty}


\subsection*{Abstract}
Testing distributed storage systems is challenging due to multiple sources of nondeterminism. Conventional testing techniques, such as unit and stress testing, are ineffective in preventing serious but subtle bugs from reaching production. Formal techniques, such as TLA+, can only verify high-level specifications of systems at the level of logic-based models, and fall short of checking the actual executable code. In this paper, we present a new methodology for testing production distributed systems. Our approach applies systematic testing techniques to thoroughly check that the executable code adheres to its high-level specifications, which significantly improves coverage of important system behaviors.

Our methodology has been applied to three production distributed storage systems in the Microsoft Azure platform. In the process, numerous bugs were identified, reproduced, confirmed and fixed. These bugs required a subtle combination of concurrency and failures, making them extremely difficult to find with conventional testing techniques. An important advantage of our approach is that a bug is uncovered in a small setting and witnessed by a full system trace, which dramatically increases the productivity of debugging.

\section{Introduction}
\label{sec:intro}

Distributed systems are notoriously hard to design, implement and test~\cite{cavage2013there, laguna2015debugging, maddox2015test}. This is due to many well-known sources of \emph{nondeterminism}~\cite{chandra2007paxos}, such as race conditions in the asynchronous interaction between system components, the use of multithreaded code inside a component, unexpected node failures, unreliable communication channels and data losses, and interaction with (human) clients.
%
All these sources of nondeterminism translate into \emph{exponentially} many execution paths that a distributed system might potentially execute. A bug might hide deep inside one of these paths and only manifest under extremely rare corner cases~\cite{gray1986computers, musuvathi2008finding}.

Classic techniques that product groups employ to test, verify and validate their systems (such as code reviews, unit testing, stress testing and fault injection) are unable to capture and control all the aforementioned sources of nondeterminism, which causes the most tricky bugs being missed during testing and only getting exposed after a system has been put in production. Discovering and fixing bugs in production, though, is bad for business as it can cost a lot of money and many dissatisfied customers.

We interviewed engineers from the Microsoft Azure team regarding the top problems in distributed system development, and the unified response was that the most critical problem today is how to improve \emph{testing coverage} to find bugs \emph{before} a system goes in production. The need for better testing techniques is not specific to Microsoft; other companies, such as Amazon and Google, publicly acknowledge~\cite{newcombe2015aws} that testing methodologies have to improve to be able to reason about the correctness of increasingly more complex distributed systems.

Amazon recently published an article~\cite{newcombe2015aws} that describes their use of TLA+~\cite{lamport1994temporal} to detect distributed system bugs and prevent them from reaching production. TLA+ is a powerful specification language for verifying distributed protocols, but it is unable to verify the code that is actually being executed. The implied assumption is that a model of the system will be verified, and then the programmers are responsible to match what was verified with the source code of the real system. Although many design bugs can be caught with this approach, there is \emph{no guarantee} that the real distributed system will be free of bugs.

In this work, our goal is to \emph{test what is being executed}. We present a new methodology for testing legacy distributed systems and uncovering bugs before these systems are released in the wild. We achieve this using \psharp~\cite{deligiannis2015psharp}, an extension of the mainstream language \csharp that provides two key capabilities: (i) a flexible way of modeling the environment using simple language features; and (ii) a systematic concurrency testing framework that is able to capture and take control of all the nondeterminism in a real system (together with its modeled environment) and systematically explore execution paths to discover bugs.

We present three case studies of using \psharp to test production distributed systems for Windows Azure inside Microsoft: a distributed storage management system and a live migration protocol. Using \psharp, we managed to uncover a very subtle bug that was haunting developers for a long time as they did not have an effective way to reproduce the bug and nail down the culprit. \psharp uncovered this bug in a very small setting, which made it easy to examine traces, identify and eventually fix the problem.

To summarize, our contributions are as follows:

\begin{itemize}
\item We present a methodology that allows flexible modeling of the environment of a distributed system using simple language mechanisms.
\item Our infrastructure can test production code written in \csharp, which is a mainstream language.
\item We present three case studies of using \psharp to test production distributed systems, finding bugs that could not be found with traditional testing techniques.
\item ...
\end{itemize}


\section{Motivating Example}
\label{sec:motivation}

Microsoft Azure Storage is a cloud storage system that provides customers the ability to store seemingly limitless amounts of data. It has grown from 10s of Petabytes (PB) in 2010 to Exabytes (EB) in 2015, with the total number of objects stored well-exceeding 60 trillion.

Azure Storage vNext is the next generation storage system for Microsoft Azure, where the primary design target is to increase the scalability by 100x or more. Similar to the current system, vNext employs containers, called {\em extents}, to store data. Extents are typically several Gigabytes each, consisting of many data blocks, and replicated over multiple {\em Extent Nodes} (ENs). In contrast to the current system, which employs a Paxos-based, centralized mapping from extents to ENs, vNext achieves its scalability target by employing a completely distributed mapping. In vNext, extents are divided into partitions, with each partition managed by a light-weight {\em Extent Manager} (ExtMgr).

\begin{figure}[t]
\centering
\includegraphics[width=\linewidth]{img/azurestore}
\caption{Top-level components of a distributed extent management system for Windows Azure.}
\label{fig:azurestore}
\end{figure}

One of the many responsibilities of ExtMgr is to ensure that every extent maintains enough replicas in the system. To achieve this, ExtMgr receives frequent periodic \emph{heartbeat} messages from every EN. Failure of EN is detected by missing heartbeats. ExtMgr also receives less frequent, but still periodic {\em synchronization reports} from every EN. The sync reports list all the extents (and associated metadata) stored on the EN. Based on these two types of messages, ExtMgr identifies which ENs have failed and which extents are affected and missing replicas. ExtMgr, then, schedules tasks to repair the affected extents and distributes the tasks to ENs. ENs repair the extents from their existing replicas in the system and lazily update ExtMgr via future sync reports.

To ensure correctness, the developers of vNext have instrumented extensive, multiple levels of testing for Extent Manager:
\begin{enumerate}
\item \emph{Unit testing}, which sends emulated heartbeats and sync reports to ExtMgr and verifies that the messages are processed as expected.

\item \emph{Integration testing}, which launches an ExtMgr together with multiple ENs, subsequently injects an EN failure, and finally verifies that the affected extents are eventually repaired.

\item \emph{Stress testing}, which launches an ExtMgr with multiple ENs and multiple extents. It keeps repeating the following process: injecting an EN failure, launching a new EN and verifying that the affected extents are eventually repaired.
\end{enumerate}

Despite of the extensive testing efforts, the vNext developers have been plagued by what appears to be an elusive bug in ExtMgr. All the unit test and integration test suites successfully pass every single time. However, the stress test suite could fail {\em from time to time} after very long executions, manifested as the replicas of some extents remain missing while never being repaired. The bug appears difficult to identify, reproduce and troubleshoot. First, it takes very long executions to trigger. Second, an extent not being repaired is {\em not} a property that can be easily verified. In practice, the developers rely on very large time-out period to detect the bug. Finally, by the time that the bug is detected, very long execution traces have been collected, which makes manual inspection tedious and ineffective.

To uncover this bug and many other similar ones, the developers are in constant search of a generic and systematic approach for testing distributed storage systems.

%Since this is a rather typical dilemma in the development of distributed storage systems, a general and systematic approach hopefully would be very helpful to the developers and greatly increase their productivity.


\section{Our Approach to Testing Distributed Systems}
\label{sec:overview}

Distributed systems typically consist of two or more components that communicate \emph{asynchronously} by sending and receiving messages through a network layer~\cite{lamport1978time}. Each component has its own input message queue, and when a message arrives, the component responds by executing an appropriate \emph{message handler}. Such a handler consists of a sequence of program statements that might update the internal state of the component, send a message to another component in the system, or even create an entirely new component.

\subsection{Challenges in testing}
\label{sec:bg:challenges}

In a distributed system, message handlers can interleave in arbitrary order, because of the asynchronous nature of message-based communication. To complicate matters further, unexpected failures are the norm in production systems: nodes in a cluster might fail at any moment, and thus programmers have to implement sophisticated mechanisms that can deal with these failures and recover the state of the system. Moreover, with multicore machines having become a commodity, individual components of a distributed system are commonly implemented using multithreaded code, which adds another source of nondeterminism.

All the above sources of nondeterminism (as well as nondeterminism due to timeouts, message losses and client requests) can easily create \emph{heisenbugs}~\cite{gray1986computers, musuvathi2008finding}, which are corner-case bugs that are difficult to detect, diagnose and fix, without using advanced \emph{asynchrony-aware} testing techniques. Techniques such as unit testing, integration testing and stress testing are heavily used in industry today for finding bugs in production code. However, these techniques are not effective for testing distributed systems, as they are not able to capture and control the many sources of nondeterminism.

The ideal testing technique should be able to work on unmodified distributed systems, capture and control all possible sources of nondeterminism, systematically inject faults in the right places, and explore all feasible execution paths. However, this is easier said than done when testing production systems.

A completely different approach for reasoning about the correctness of distributed systems is to use formal methods.  A notable example is TLA+~\cite{lamport1994temporal}, a formal specification language that can be used to design and verify concurrent programs via model checking. Amazon recently published an article describing their use of TLA+ in Amazon Web Services to verify distributed protocols~\cite{newcombe2015aws}. A limitation of TLA+, as well as other similar specification languages, is that they are applied on a model of the system and not the actual system. Even if the model is verified, there is no guarantee that the code that will actually execute is free of bugs. \SCComment{replace the last sentence with "the gap between a real-world implementation and the verified model is still significant, so implementation bugs are still a realistic concern." ?}

\subsection{Types of bugs}
\label{sec:bg:bugs}

We can classify most distributed system bugs in two categories: \emph{safety} and \emph{liveness} property violations~\cite{lamport1977proving}.

\begin{description}
\item[Safety] A safety property checks that an erroneous program state is \emph{never} reached, and is satisfied if it \emph{always} holds in each possible program execution.

\item[Liveness] A liveness property checks that some progress \emph{will} happen, and is satisfied if it \emph{always eventually} holds in each possible program execution.
\end{description}

\noindent
A safety property can be specified using an \emph{assertion} that fails if the property gets violated in some program state. An example of a generic safety property for message passing systems is to assert that whenever a message gets dequeued there must be an action that can handle the received message.

Liveness properties are much harder to specify and check since they apply over entire program executions and not just individual program states. Normally, liveness checking requires the identification of an infinite fair execution that never satisfies the liveness property~\cite{schuppan2004efficient, musuvathi2008fair}. Prior work~\cite{schuppan2004efficient} has proposed that assuming a program with finite state space, a liveness property can be converted into a safety property. Other researchers proposed the use of heuristics and only exploring finite executions of an infinite state space system using random walks to identify if a liveness property is violated~\cite{killian2007life}.


\section{Testing Azure Storage vNext with \psharp}
\label{sec:method}

Our goal in this work is to \emph{test what is being executed}. Our approach involves using \psharp~\cite{deligiannis2015psharp}, a framework that provides: (i) an \emph{event-driven asynchronous programming} language for developing and modeling distributed systems; and (ii) a \emph{systematic concurrency testing} engine that can systematically explore all interleavings between asynchronous event handlers, as well as other nondeterministic events such as failures and timeouts.

\subsection{The \psharp framework}
\label{sec:method:psharp}

The \psharp language is an extension of \csharp, built on top of Microsoft's Roslyn\footnote{\url{https://github.com/dotnet/roslyn}} compiler, that enables asynchronous programming using communicating state-machines. \psharp machines can interact asynchronously by sending and receiving events,\footnote{We use the word ``event'' and ``message'' interchangeably.} an approach commonly used to develop distributed systems. This programming model is similar to actor-based approaches provided by other asynchronous programming languages (e.g. Scala~\cite{odersky2008programming} and Erlang~\cite{armstrong1996erlang}).

A \psharp machine consists of an input event queue, states, state transitions, event handlers, fields and methods. Machines run concurrently with each other, each executing an event handling loop that dequeues an event from the input queue and handles it by invoking an appropriate event handler. This handler might update a field, create a new machine, or send an event to another machine. In \psharp, a send operation is non-blocking; the message is simply enqueued into the input queue of the target machine, and it is up to the operating system scheduler to decide when to dequeue an event and handle it. All this functionality is provided in a lightweight runtime library, build on top of Microsoft's Task Parallel Library~\cite{leijen2009tpl}.

Because \psharp is built on top of \csharp, the programmer can blend \psharp and \csharp code; this not only lowers the overhead of learning a new language, but also allows \psharp to easily integrate with legacy code. Another advantage is that the programmer can use the familiar programming and debugging environment of Visual Studio.

A key capability of the \psharp runtime is that it can run in \emph{bug-finding mode}, where a embedded systematic testing engine captures and takes control of all sources of nondeterminism (such as event handler interleavings, failures, and client requests) in a \psharp program, and then systematically explores all possible executions to discover bugs.

\psharp is available as open-source\footnote{\url{https://github.com/p-org/PSharp}} and is currently used by various teams in Microsoft to develop and test distributed protocols and systems.

%The \psharp language belongs to the same family of languages as P~\cite{desai2013p}.

\subsection{Overview of our approach}
\label{sec:method:model}

In previous work~\cite{deligiannis2015psharp}, we approached the problem of testing legacy distributed systems as follows. First, we ported the system to \psharp, then we modeled its environment as \psharp state machines, and finally we tested the ported system and its environmental model using the \psharp systematic concurrency testing engine. The limitation of this approach is that it does not allow us to directly test a legacy system, as it has to be re-implemented first in \psharp. However, such endeavor is very costly and time consuming, and thus is not realistic for testing an existing production system, such as the Azure Storage vNext. Also, unless the code under test is the one that will actually execute, there is no guarantee that the real system will be bug-free.

To solve this problem, and allow \psharp to be used for testing legacy distributed systems, we decided to take a radically different approach. We provide the capability to model the environment of a system using \psharp, and then allow the developer to take advantage of existing language features, such as \emph{method dispatch}, to connect the system under test with the environmental model, and finally test it using the \psharp systematic concurrency testing engine.

We argue that our approach is \emph{flexible} since it allows the user to model \emph{as much} or \emph{as little} of the environment as required to achieve the desired level of testing. We also argue that our approach is \emph{generic} since a programmer can build on top of it to test more complicated use cases (see Section~\ref{}). Furthermore, the language features that are required to be used to connect the real code with the modeled code, are already being heavily used in production for testing purposes, which makes this method approachable to product groups.

\subsection{Modeling the environment}
\label{sec:method:model}

The environment of a distributed system might consist of other distributed systems and services, clients, operating system timers, as well as libraries for networking or other purposes. To be able to systematically test a distributed system, this environment must be modeled and all the interactions between the environment and the system, as well as all the nondeterminism, must be captured and controlled by the \psharp runtime.

\begin{figure}[t]
\centering
\includegraphics[width=\linewidth]{img/mocked_engine}
\caption{The real environment of the Extent Manager is replaced with a mocked version for testing.}
\label{fig:azurestoremodel}
\end{figure}

\subsubsection{Using method dispatch for modeling}
\label{sec:method:model:dd}

Method dispatch is the process of selecting which method, from a set of available methods with the same interface, should be invoked during a program's execution. There are two types of method dispatch: \emph{static}, which is resolved during compilation; and \emph{dynamic}, which is resolved in runtime.  \csharp (and thus \psharp) supports both static and dynamic dispatch, and provide the \texttt{virtual} modifier that can be used to declare a method which can be \emph{overridden} during runtime by an inheriting class. This capability is provided by the common language runtime (CLR) of Microsoft's .NET framework, and is a key feature of \csharp as well as other mainstream object-oriented languages.

Using method dispatch for modeling is straightforward. The system under test exposes a set of APIs as \emph{virtual methods}. The developer can then \emph{override} these APIs and replace them with \emph{mocks} that will execute instead of the original implementations during systematic testing with \psharp.

\begin{figure}[t]
\begin{lstlisting}
// Public interface of the real network engine
class NetworkEngine {
  public virtual void SendMessage(Socket s, Message msg);
  public virtual void EnqueueMessage(Message msg);
}

// The mocked network engine used during testing
class MockedNetEngine : NetworkEngine {
  ExtentManager EM; // Handle to actual system under test
  MachineId Env; // Handle to modeled environment
  
  public MockedNetEngine(ExtentManager em, MachineId env) {
    this.EM = em;
    this.Env = env;
  }
  
  public override void SendMessage(Socket s, Message msg) {
    PSharpRuntime.Send(this.Env, new MsgEvent(), s, msg);
  }
  
  public override void EnqueueMessage(Message msg) {
    this.EM.ProcessMessage(msg);
  }
}
\end{lstlisting}
\vspace{-2mm}
\caption{The mocked network engine used for testing the Azure Storage vNext system.}
\label{fig:enginecode}
%\vspace{-2mm}
\end{figure}

We now give an example of using dynamic dispatch to model the network engine of an extent manager in the Azure Storage vNext case study (see Figure~\ref{fig:enginecode}). The network engine is responsible for sending to and receiving messages from the various components of the system. During real execution, the network engine uses a custom remote procedure call (RPC) .NET library for communication. For testing, though, it is desirable to replace all calls to this RPC library with \psharp send and receive operations, which can be captured and systematically interleaved to find bugs. We easily achieved this by exposing the original send message operation of the network engine as a virtual method, and then overriding it for testing. In the overridden method, we created a \psharp event and then we wrapped the original message in this event's payload. Then, instead of invoking the RPC library, we invoke the \texttt{PSharpRuntime.Send(...)} method, which asynchronously sends the event (containing the original message) to the target extent node machine.

For mocking the receive operation, we take advantage of the implicit receive of events in \psharp machines. When a extent node machine receives an event, an appropriate event handler is invoked, which extracts the original message from the payload and then handles it accordingly.

\subsubsection{Abstracting timers}
\label{sec:method:model:timers}

Distributed systems are often using timers to determine when an event should be send from one component to another. For example, in the Azure Storage vNext system, each Extent Node is associated with a timer that fires of a synchronization message every 5 minutes and a heartbeat every 5 seconds. This timer is related to the liveness bug that we discovered: the synchronization message that gets fired every 5 minutes can potentially race with an Extent Node failure; if it arrives after the node failed, then the bug would manifest. Traditional testing techniques cannot easily find such a bug, due to the very infrequent occurrence of this race due to the timer. 

Our methodology in \psharp to systematically test distributed systems that rely on timers, is to abstract timers away, model them using message passing communication and introduce nondeterminism in their firing.

Figure~\ref{fig:timer} shows how we modeled a generic timer in the Azure Storage vNext case study. The Extend Manager, as well as each Extent Node in the harness, is associated with a unique \texttt{Timer} machine. When creating this machine, we pass as a payload the id of the machine that owns this timer. When the \texttt{Timer} machine is created, it stores this id in the \texttt{Owner} field and then transitions to the \texttt{Active} state. In this state, the \texttt{Timer} loops infinitely and nondeterministically sends a \texttt{TimerTickEvent} to \texttt{this.Owner}. When the Extent Node owner receives this event, it handles it by generating a synchronization message that is being send to the Extent Manager. Similarly, when the Extent Manager receives a \texttt{TimerTickEvent} from its own \texttt{Timer}, it handles it by nondeterministically invoking repair-related methods in the Extent Repair Center data structure. 

\begin{figure}[t]
\begin{lstlisting}
internal class Timer : Machine
{
  MachineId Owner; // Id of the owner machine

  [Start]
  [OnEntry(nameof(InitOnEntryAction))]
  [OnEventGotoState(typeof(Unit), typeof(Active))]
  class Init : MachineState { }

  void InitOnEntryAction()
  {
    this.Owner = (MachineId)this.Payload;
    // triggers state transition to Active
    this.Raise(new Unit());
  }

  [OnEntry(nameof(ProcessTickEvent))]
  [OnEventGotoState(typeof(Unit), typeof(Active))]
  class Active : MachineState { }

  void ProcessTickEvent()
  {
    // Nondeterministic boolean choice controlled by P#
    if (this.Nondet())
      // sends a timer tick event to the owner machine
      this.Send(this.Owner, new TimerTickEvent());
    // triggers state transition to Active
    this.Raise(new Unit());
  }
}
\end{lstlisting}
\vspace{-2mm}
\caption{Timers in Azure Storage vNext are modeled as a nondeterministic \psharp machines.}
\label{fig:timer}
%\vspace{-2mm}
\end{figure}

\subsubsection{Modeling and injecting failures}
\label{sec:method:model:failures}

In production, each Extent Node of the Azure Storage vNext system periodically (every 5 seconds) sends a heartbeat to the Extent Manager which notifies that the Extent Node is alive. Because we want to model failures and systematically inject them using \psharp, we abstract away the heartbeat mechanism in our harness. However, the Extent Manager logic relies on time intervals to detect node failures (see Figure~\ref{fig:expiration}). The way to abstract this time-related logic and connect the real code with the modeled code is to use virtual dispatch and override the virtual \texttt{IsNodeExpired} method with a mocked version.

\begin{figure}[t]
\begin{lstlisting}
// Real code for detecting node expiration
public virtual bool IsNodeExpired(string node_id, DateTime expiration)
{
  return DateTime.Compare(expiration, DateTime.Now) <= 0;
}

// Mocked code for detecting node expiration
public override bool IsNodeExpired(string node_id, DateTime expiration)
{
  return this.DeletedNodes.Contains(node_id);
}
\end{lstlisting}
\vspace{-2mm}
\caption{Abstracting the node expiration logic in the Extent Manager component of Azure Storage vNext.}
\label{fig:expiration}
%\vspace{-2mm}
\end{figure}

Figure~\ref{fig:expiration} presents how we mocked the node expiration detection method. Instead of comparing the time interval as in the original code, we now check if the set \texttt{DeletedNodes} contains the id of the Extent Node that we are checking for expiration. If it contains the id, then it means that the node has failed. The \texttt{Environment} machine that we have created as part of our testing \psharp harness, will nondeterministically choose a node to kill, then send the id of this killed node to the Extent Manager wrapper machine, who will in turn add it to the \texttt{DeletedNodes} set.

\subsection{Handling intra-machine concurrency}
\label{sec:method:async}

Async/Await

custom schedulers, etc

\subsection{Other}
\label{sec:method:other}

Logs/traces -> user can extend them?

\PDComment{mention dependency injection pattern?}


\section{Additional Case Studies}
\label{sec:cases}

The modeling and testing approach described in earlier sections of the paper is not specific to the vNext system. \psharp is a generic framework, capable of handling arbitrary distributed systems. We showcase this capability by presenting developer experience in using \psharp to model and test two other distributed storage systems used in production in Microsoft: the Live Azure Table Migration library; and the Azure Service Fabric system.

\subsection{Live Azure Table Migration}
\label{sec:cases:migration}

The Live Azure Table Migration (MigratingTable) is a library for
\emph{transparently migrating} a data set between tables in an Azure Storage
service \emph{while} an application is accessing this data set. This case study
is interesting because its developers built the \psharp model of the system
while developing the library to increase confidence in its implementation;
indeed they discovered many bugs during this \emph{co-development} process (see
Section~\ref{}). Another interesting aspect of this system is that it heavily
uses the \texttt{async}/\texttt{await} primitives, which were not supported in
the original \psharp runtime, and thus required us to extend \psharp with
support for intra-machine concurrency (see Section~\ref{}). Finally, multiple 
complex safety properties were written to test MigratingTable with \psharp, 
whereas the vNext model focused on a single liveness property.

\begin{figure}[t]
\centering
\includegraphics[width=\linewidth]{img/mocked_migration}
\caption{Environmental model of MigratingTable (each box with a dotted line represents one \psharp machine).}
\label{fig:mockedmigration}
\end{figure}

MigratingTable provides a \emph{virtual table} that has a similar interface to an ordinary Azure table. This interface is named \texttt{IChainTable}. The virtual table is backed by a pair of \emph{old} and \emph{new} tables. A background \emph{migrator} job is responsible for moving all data from the old table to the new table. Meanwhile, each read and write operation issued to the virtual table is translated to a sequence of reads and writes on the backend tables according to a protocol that guarantees linearizability of operations on the virtual table across multiple application processes, assuming that the backend tables respect their own linearizability guarantees.

The testing goal was to ensure that when multiple application processes, implementing the \texttt{IChainTable} interface, issue read and write operations to their own MigratingTable instances with the same backend tables, the behavior complies with the specification of \texttt{IChainTable} for the combined \emph{input history}.

%\begin{figure}[t]
%\centering
%\includegraphics[width=\linewidth]{img/livemigration}
%\caption{Resharding a data set when a third Azure storage account is added. Two key ranges are each migrated to the new account using a MigratingTable instance (abbreviated MTable).}
%\label{fig:livemigration}
%\end{figure}

% N.B. Artifact Services is mentioned at http://research.microsoft.com/en-us/people/schulte/.  Hopefully it's OK to reveal that it was the system in this case study. ~ Matt 2015-08-17
%The initial motivation for MigratingTable was to solve a scaling problem for Artifact Services, an internal Microsoft system with a data set that is sharded across tables in different Azure storage accounts because it exceeds the limit on traffic supported by a single Azure storage account.  As the traffic continues to grow over time, the system needs to reshard the data set across a greater number of Azure storage accounts without interrupting service.  During such a resharding, our sharding manager will identify each key range that should migrate to a different table, and we will use a separate MigratingTable instance for each such key range to actually perform the migration (Figure~\ref{fig:livemigration}).  MigratingTable may also be useful to migrate data to a table with different values of configuration parameters that Azure does not support changing on an existing table, such as geographic location.

%Since we were designing a new concurrent protocol that we expected to become increasingly complex over time as we add optimizations, we planned from the beginning to maintain a \psharp test harness along with the protocol to maintain confidence in its correctness.

%MigratingTable implements an interface called \texttt{IChainTable}, which provides the core read and write functionality of the original Azure table API with one exception: it provides \emph{streaming reads} with a weaker consistency property than multi-page reads in the original API, since the original property would have been difficult to achieve for no benefit to applications we could foresee.  MigratingTable requires that its backend tables also implement \texttt{IChainTable}, and we wrote a simple adapter to expose physical Azure tables as \texttt{IChainTable}.

% N.B. \texttt{SpecTable} = InMemoryTableWithHistory in the current codebase. ~ Matt 2015-08-17

The main challenge behind testing MigratingTable is that there are many possible input histories and that the system is highly concurrent. The developers could have tested specific input histories, but they were not confident that this approach would be effective in catching bugs, especially because concurrency increases the potential for difficult to foresee interactions between different parts of the code.

\subsubsection{Testing the MigratingTable library}

Towards testing the MigratingTable library, the developers wrote an in-memory reference implementation of the \texttt{IChainTable} interface, called \texttt{SpecTable}, which can be used for comparing the output of MigratingTable on an arbitrary input history. This enabled sampling from a distribution, which was defined over all possible input histories within certain bounds. \psharp takes control of the choice of input history, so that it can systematically explore different input scenarios.

%\psharp takes control of the choice of input history, as well as the schedule, so both can be reproduced using a single random seed. Then, under the \emph{small scope hypothesis} that any bug in MigratingTable leads to incorrect output for at least one input history in our distribution, we have a positive probability of detecting this incorrect output on each iteration of the \psharp test.

%If we had no formalization of the specification and had to rely on expected outputs worked out by hand, this might be the best we could do.  However, since the \texttt{IChainTable} specification is relatively simple and is almost deterministic under sequential calls, it was straightforward to write an in-memory reference implementation called \texttt{SpecTable} to which we can compare the output of MigratingTable on an arbitrary input history.  This gave us the attractive option to sample from a distribution we defined over all possible input histories within certain bounds.

%It was convenient to let \psharp control the choice of input history as well as the schedule so we could reproduce both using a single random seed.  Then, under the \emph{small scope hypothesis} that any bug in MigratingTable leads to incorrect output for at least one input history in our distribution, we have a positive probability of detecting this incorrect output on each iteration of the \psharp test.

%All of our input histories include two application processes.  Each process performs either a single streaming read or a sequence of two atomic calls, each a read or a batch write.  Each batch write call includes one or two operations, where the operation type is chosen from the set supported by \texttt{IChainTable} (Insert, Replace, Merge, Delete, InsertOrReplace, InsertOrMerge, DeleteIfExists) and the row key is chosen from $\{0, \ldots, 5\}$.  If the operation requires an If-Match value, it is equally likely to be \texttt{*}, the current ETag of the row (if it exists), or some non-matching value.  Finally, the new entity includes a user-defined property \texttt{isHappy} whose value is equally likely to be true or false.  For both atomic and streaming reads, the filter expression is equally likely to be empty (i.e., match everything), \texttt{isHappy eq true}, or \texttt{isHappy eq false}.

%As mentioned above, the \texttt{IChainTable} specification is almost deterministic under sequential calls; the only nondeterminism is in the results of streaming reads.  Given a streaming read, \texttt{SpecTable} can compute the set of all results that are compliant with the specification, so we can simply check if the result of MigratingTable is in this set.

%To test MigratingTable, we must supply it with backing tables.  We use \texttt{SpecTable} for this purpose as well, with \psharp choosing the actual result of each streaming read from the valid set.  Our correctness property is then:
% Convert to some theorem-like environment? ~ Matt
%\begin{quote}
%For every execution trace of a collection of MigratingTables backed by the same pair of \emph{old} and \emph{new} \texttt{SpecTable}s in parallel with the migrator job, there exists a linearization of the combined input history such that the output in the original trace matches the output of a ``reference'' \texttt{SpecTable} on the linearized input.
%\end{quote}
%

%The MigratingTable was instrumented to report the intended \emph{linearization point} of each input call, which in our setting is always one of the corresponding \emph{backend calls} to the backend tables (often the last).  Specifically, after each backend call completes, MigratingTable reports whether that call was the linearization point, which may depend on the result of the call.  This makes it possible to check the correctness property as the model executes.

The MigratingTable was instrumented to report the intended \emph{linearization point} of each input call. Specifically, after each input call completes, MigratingTable reports whether that call was the linearization point, which may depend on the result of the call.  This makes it possible to check the correctness property as the model executes.

The \psharp environmental model of MigratingTable consists of a \texttt{Tables} machine containing the old, new and reference table implementations; a collection of \texttt{Service} machines containing identically configured MigratingTables; and a \texttt{Migrator} machine that performs the background migration (see Figure~\ref{fig:mockedmigration}).

Each \texttt{Service} machine issues a random sequence of input calls to its MigratingTable, which sends backend calls to the \texttt{Tables} machine. When MigratingTable reports the linearization point of an input call, the \texttt{Service} machine sends that input call to the reference table.  When an input call completes, the \texttt{Service} machine checks that the results from the MigratingTable and the reference table agree.

\psharp captures and controls the interleaving of the backend calls. To ensure that the reference table is never observed to be out of sync with the backend tables, after the \texttt{Tables} machine processes a backend call, it enters a state that defers further backend calls until MigratingTable has reported whether the backend call was a linearization point and (if so) the call to the reference table has been made.

%We use the \psharp random scheduling strategy; we were afraid that an exhaustive strategy would only be feasible within bounds so low that we would miss some bugs.

%We wanted to implement the core MigratingTable algorithms in \csharp ``async/await'' code, like most of Artifact Services, to achieve both good readability and good performance.  We used a method similar to that described in Section~\ref{sec:psharp:async} to bring the generated TPL tasks under the control of the \psharp scheduler.  Then we implemented an ``async'' RPC mechanism based on the .NET RealProxy class that automates the generation of proxies for objects hosted by other \psharp machines (in our setting, the service machines use proxies for the \texttt{SpecTable}s and various auxiliary objects hosted by the tables machine).  When a machine calls a method on a proxy, the proxy sends a \psharp message to the host machine, causing it to execute the method call on the original object and send back the result, which the proxy then returns.  Thus, the use of these proxies as \texttt{IChainTable} backends is transparent to the MigratingTable library, thanks to dynamic dispatch.

\subsection{Azure Service Fabric}
\label{sec:cases:fabric}

- \psharp generalized beyond vNext. \psharp is not only applicable to vNext but in many systems.\\
- (Advanced) safety properties?\\
- What we developed is a model that other people can reuse. Plugin the model and test the user application.\\
- Example of advanced modeling capability of \psharp. One component talks to another sophisticated API. We demonstrate environmental modeling expressivity in system engineering.


\emph{Azure Service Fabric}\footnote{\url{http://azure.microsoft.com/en-gb/campaigns/service-fabric/}} (or \emph{Fabric} for short) is a platform and API for creating reliable services that execute on a cluster of machines. 
%The developer writes a service that receives requests (e.g.\ from some client program via HTTP requests) and mutates its state based on these requests. 
In order to make the user-written service \emph{reliable}, Fabric launches several \emph{replicas} (copies) of the service, where each replica runs as a separate process on a different node in the cluster.
%\PTComment{Cut: description of primary and secondaries, and electing a new primary.}
%One replica is selected to be the \emph{primary} which serves client requests; the rest are \emph{secondaries}. The primary replicates state changes to the secondaries by sending \emph{replication requests} so that all replicas eventually have the same state. If the primary fails (e.g.\ if the node on which the primary is running crashes), Fabric elects one of the secondaries to be the new primary and launches another secondary; the new secondary will receive a full or partial copy (depending on whether persistent storage is used) of the state of the new primary in order to ``catch up'' with the other secondaries. Fabric provides a name-resolution service so that clients can always find the current primary.
The state of the service is replicated from the \emph{primary} replica to the other \emph{secondary} replicas for redundancy.
Since user-written Fabric services are complex asynchronous and distributed applications, 
they are challenging to test.
% They are interesting targets for systematic testing with \psharp{}.

Our primary goal was to create a \psharp{} model of Fabric to allow
thorough testing of services, where Fabric's asynchrony is controlled 
by the \psharp{} runtime.
The model is written once
to include all behaviours of Fabric
so that it can be used again and again to test arbitrary Fabric services.
This was the largest among the case studies
and required multiple rounds of debugging.
The availability of systematic testing
allowed us to debug the model to
a point where reported assertion violations indicate a bug in the
user service.
Without systematic testing,
even bugs in the model would have been hard to find.
Note that we model the lowest Fabric API layer (\texttt{Fabric.dll})
which is not documented for use externally.
Eventually, we will lift the model
to higher layers
but for this paper we study internally-developed services
that target the lowest layer. 

%\PTComment{Cut: prior work created a Fabric model\ldots}
% Prior work~\cite{deligiannis2015psharp}
% created a model of Fabric with limited functionality;
% it used a mixture of \csharp{} and \psharp{} internally,
% only supported one in-flight replication request
% (which restricts the asynchrony that can be tested),
% and only supported one Fabric service.
% Our new Fabric model was re-written to use only \psharp{}
% internally,
% support an arbitrary number of Fabric services and in-flight replication requests,
% and in general be a more complete model of Fabric.
% Note that \csharp{} code is still required to interface with the existing
% \csharp{} service code.
% We refer to our \csharp{} code
% as the \emph{translation layer}
% and
%  the user-written \csharp{} service code as \emph{user code}. 

%\PTComment{Cut: Figure of model.}
% \begin{figure}[thb]
% \centering
% \includegraphics[width=\linewidth]{img/fabricmodel}
% \caption{Overview of the key machines and interfaces in our Fabric model.}
% \label{fig:fabric_model}
% \end{figure}

%\PTComment{Cut: overview of model.}
% An overview of our Fabric model is shown in Figure~\ref{fig:fabric_model}.
% The \texttt{ClusterRuntime} machine 
% handles the creation and management of 
% one or more Fabric services,
% as well as service resolution requests
% which allows for client-service and inter-service communication
% within the model.
% Each Fabric service instance is managed by a \texttt{ServiceRuntime}
% machine, which in turn manages 
% several \texttt{ReplicaRuntime} machines.
% Each \texttt{ReplicaRuntime} communicates with the user code
% via several machines and interfaces from the translation layer
% (only the translation layer for the primary is shown,
% but every \texttt{ReplicaRuntime} has its own instance of the translation layer).
% Note that communication between machines is hierachical;
% thus, communication between \texttt{ReplicaRuntime}s
% (such as the sending of replication requests)
% is via the \texttt{ServiceRuntime} machine for that service.
% This approach does not necessarily reflect how Fabric works in practice.
% Instead, we chose an architecture
% that keeps the model simple
% while still allowing (what we believe to be) realistic
% asynchrony and failure scenarios.

%\PTComment{Cut: Description of how replication works in the model, including how we modeled at a fine granularity.}
% \textbf{Replication example:} In order to replicate a state-mutating operation,
% user code at the primary replica 
% calls \texttt{IStateReplicator.ReplicateAsync},
% passing the serialized operation
% object.
% The operation is sent to the \texttt{ServiceRuntime},
% where it is assigned a \emph{logical sequence number} (LSN);
% each operation is assigned a consecutive LSN
% to track the total-order in which operations should be applied.
% The LSN is sent back to the primary replica
% where it is returned from the \texttt{ReplicateAsync} call,
% along with a \texttt{Task} object that will ``complete'' once
% the operation has been replicated to a majority of
% secondaries;
% thus, the user code can wait on the \texttt{Task}
% before confirming to any clients that the request has been applied reliably.
% The \texttt{ServiceRuntime} adds the operation to its list of in-flight
% replication requests and sends $n$ events to itself to signal that the request
% must be sent to a replica, where $n$ is the number of secondary replicas.
% The reason for sending $n$ events to itself instead of simply sending events
% directly to each secondary is so that the \texttt{ServiceRuntime}
% can process a simulated failure event inbetween the sending of replica requests
% to each secondary.
% This is an example of where we carefully considered
% the granularity of actions so that we could 
% model failures appropriately.
% The user code at a secondary receives the operation,
% applies it and then calls \texttt{Acknowledge} on the operation object;
% we implement this to send an event to the \texttt{ReplicaRuntime}
% which forwards the acknowledgement to the \texttt{ServiceRuntime}.
% Once a majority of secondaries have acknowledged, the
% \texttt{ServiceRuntime} removes the replication request from its list
% of replication requests
% and sends an acknowledgement to the primary,
% where the previously returned \texttt{Task} completes. 

%\PTComment{Cut: We reverse-engineered Fabric as needed.}
% \subsubsection{Fabric model correctness}
% Our model does not attempt to
% simulate the internals of Fabric accurately,
% as its purpose is to find bugs in user code
% and not in Fabric itself (which we assume to be correct).   
% However,
% due to lack of documentation,
% it is not always clear how Fabric should behave
% in certain scenarios.
% Thus,
% we ran several variants of a simple Fabric service
% that logs calls into the user code in order to
% reverse-engineer the actual behaviour of Fabric;
% we ensured that our model has the same behaviour,
% although this is an ongoing process as we encounter
% additional scenarios.

%\PTComment{Cut: We used systematic testing to find bugs in our model!}
% A further problem is that our model may contain bugs.
% In order to find bugs in our model effectively,
% we wrote a \psharp{} service made up of a single machine
% which takes the place
% of the user code and translation layer in Figure \ref{fig:fabric_model},
% for each \texttt{ReplicaRuntime}.
% Thus, we were able to run this pure \psharp{} system
% under \psharp{}'s systematic testing mode
% and uncover many assertion failures within our model.
% We tested a scenario where the primary fails at some non-deterministic point
% within the execution.

%\PTComment{Cut: Example bug found in our model plus the fix.}
% \textbf{Example Fabric model assertion failure:}
% In the buggy trace,
% the \texttt{ServiceRuntime}
% sends an \texttt{EEpochInfo} event to the second
% \texttt{ReplicaRuntime}
% indicating that this is the first epoch and the
% replica will be a secondary.
% An \emph{epoch} represents a configuration of primary and
% secondary replicas; when a different replica becomes the primary,
% this indicates the start of a new epoch.
% The \texttt{ReplicaRuntime} acknowledges that it has become
% a secondary by responding with the same event type.
% The \psharp{} service sends an \texttt{ESecondaryCopyContextOp};
% this indicates what state the secondary has
% and, thus, what the primary should send to this secondary so that it can catch
% up.
% The \texttt{ESecondaryCopyContextOp} event is forwarded to the
% \texttt{ServiceRuntime}. 
% The \texttt{ServiceRuntime} then receives and handles an \texttt{EKillPrimary}
% event, which causes the second replica to become the new primary.
% Thus,
% the \texttt{ServiceRuntime} sends another \texttt{EEpochInfo}
% to the second \texttt{ReplicaRuntime}
% indicating that this is the second epoch
% and the replica will be a primary.
% As part of this change,
% the \texttt{ReplicaRuntime}
% sends an event to the \psharp{} service
% indicating that it should stop waiting for
% the state to be copied from the old primary to this replica,
% which is acknowledged by sending an event to the \texttt{ReplicaRuntime}.
% This event causes the \texttt{ReplicaRuntime}
% to send an \texttt{ESecondaryCopyStateDone}
% event to the \texttt{ServiceRuntime},
% which unfortunately responds with an event indicating that the
% \texttt{ReplicaRuntime} is now an \emph{active} secondary
% (i.e.\ a secondary that has caught up with the primary).
% However, this causes an assertion failure because
% the \texttt{ReplicaRuntime} is becoming a primary and, thus,
% cannot be a secondary.
% Our fix to this bug was to ensure that the 
% \texttt{ESecondaryCopyStateDone} event was marked as part of the first
% epoch (as the \texttt{ReplicaRuntime} had not yet acknowledged the
% change to primary); thus, 
% the \texttt{ServiceRuntime} ignores the event and does not try to make the
% replica an active secondary.

% \subsubsection{CScale}

The main system that we tested is \emph{CScale}~\cite{X},
a big data-stream processing system that chains multiple Fabric services
in a directed acyclic graph.
A key challenge was that
CScale contains inter-service communication.
%\PTComment{Cut: detail.}
% that resolves services
% via Fabric but communicates
% using a non-Fabric remote procedure call protocol.
To close the system,
%\PTComment{Cut: we implemented service resolution}
% we first implemented service resolution in our Fabric model,
% which resolves Fabric service names to IP addresses.
%We then replaced 
we 
replaced the relevant classes
% replaced the inter-service communication classes with our own,
% ;
% our replacement includes the ability to map
% URLs to \psharp{} machines
so that remote procedure calls are implemented
by sending and receiving \psharp{} events.
Thus,
we converted a distributed system
that uses both Fabric and its own
network communication protocol
% that runs on the Fabric platform
% and 
% uses network communication
into 
a closed single process system. 
% that uses multiple threads,
% contains
% the Fabric model
% and the CScale services,
% and does not use network communication.
%\PTComment{Cut: we had to remove static fields.}
% Additional changes were needed to
% remove certain static (per-process) fields,
% as these were inadvertently
% being shared between services
% after making the system a single process.
A key challenge in our work
was to test CScale despite the fact that it
uses various synchronous and asynchronous APIs
other than \psharp{}.
This work is still in-progress.
However,
we were able to find a \texttt{NullReferenceException}
bug
in CScale
by running it on our model.













\section{Quantifying the Cost of Using \psharp}
\label{sec:eval}

We report our experience of applying \psharp on the three case studies discussed in this paper. We aim to answer the two following questions:

\begin{enumerate}
\item How much human effort was spent in modeling the environment of a distributed system using \psharp?

\item How much computational time was spent in systematically testing a distributed system using \psharp?
\end{enumerate}

\subsection{Cost of environmental modeling}
\label{sec:eval:human_cost}

\newcommand{\colspacing}{\hspace{1.8em}}
\begin{table}[t]
\small
\centering
\setlength{\tabcolsep}{0.3em}
\begin{tabular}{l rrrrr rr}
\centering
& \multicolumn{2}{c}{\textbf{\#LoC}}
& \multicolumn{3}{c}{\textbf{\psharp statistics}}
& \multirow{2}{*}{\textbf{\#Bugs}}\\
\cmidrule(lr){2-3}
\cmidrule(lr){4-6}

\textbf{Azure Systems}
& \multicolumn{1}{r}{\textbf{Real}}
& \multicolumn{1}{r}{\textbf{Model}}
& \textbf{\#M}
& \textbf{\#ST}
& \textbf{\#AB}
& \textbf{found}\\[0.3em]

\toprule

Azure Storage vNext
& \multicolumn{1}{r}{0}
& \multicolumn{1}{r}{684}
& \multicolumn{1}{r}{5}
& \multicolumn{1}{r}{11}
& \multicolumn{1}{r}{17}
& \multicolumn{1}{r}{1}\\

Live Table Migration
& \multicolumn{1}{r}{0}
& \multicolumn{1}{r}{0}
& \multicolumn{1}{r}{0}
& \multicolumn{1}{r}{0}
& \multicolumn{1}{r}{0}
& \multicolumn{1}{r}{$>$10}\\

Fabric User Service
& \multicolumn{1}{r}{0}
& \multicolumn{1}{r}{0}
& \multicolumn{1}{r}{0}
& \multicolumn{1}{r}{0}
& \multicolumn{1}{r}{0}
& \multicolumn{1}{r}{0}\\

\bottomrule
\end{tabular}
\caption{Statistics from modeling the environment of the three Microsoft Azure-based systems under test.}
\label{tab:stats}
%\vspace{-3mm}
\end{table}

Environmental modeling is a core activity of using \psharp. It is required for \emph{closing the environment} of a system under test and making it amenable to systematic testing. Table~\ref{tab:stats} presents program statistics for our three case studies. We report: lines of code for the system under test (\#LoC); number of bugs found in the system (\#B); lines of \psharp code for the environmental model (\#LoC); number of machines (\#M); number of state transitions (\#ST); and number of action handlers (\#AH).

Modeling the environment of the Extent Manager in the Azure Storage vNext system required approximately 2 weeks of part-time developing. The \psharp model for testing this system is the smallest (in lines of code) from all three case studies. This was because the modeling effort was targeting the particular liveness bug that was haunting the developers of vNext. We are currently in the process of modeling other components of vNext, such as a ChainReplication and a Paxos system.

Modeling the Live Migration Table required \PDComment{waiting confirmation from Matt}. This case study is interesting because the development of the actual system and its \psharp environmental model occurred side-by-side. This is in contrast with the other two case studies discussed in this paper, where the modeling activity occurred independently and at a later stage of the development process.

Modeling Fabric required approximately 4-5 months. Although this is a significant amount of time, it is a one time effort activity. Our plan is to reuse the developed Fabric model for testing arbitrary user services built for the Azure Service Fabric system.

\subsection{Cost of systematic testing}
\label{sec:eval:machine_cost}

We then present runtime results of using \psharp with two different systematic testing schedulers to find bugs in the case studies.

After we fixed all the bugs we could find in the programs under test, we added options to the code to introduce various benchmark bugs (some that were actually in the original programs and some we made up) one at a time so we could evaluate different methods to detect and diagnose them.

After fixing all the bugs we found in MigratingTable, we added an option to conditionally reintroduce each of the following bugs.

We performed all experiments using the Windows PowerShell tool on a 2.50GHz Intel Core i5-4300U CPU with 8GB RAM running Windows 10 Pro 64-bit.

\setlength{\tabcolsep}{.72em}
\begin{table*}[t]
\small
\centering
\begin{tabular}{rl rrr rrr}
\centering
& \multicolumn{2}{c}{\textbf{Stress Testing}}
& \multicolumn{5}{c}{\textbf{\psharp Testing (Random Scheduler)}}\\
\cmidrule(lr){2-3}
\cmidrule(lr){4-8}


&
& \multirow{1}{*}{\textbf{Bug}}
& \multirow{1}{*}{\textbf{Time to}}
& \multirow{1}{*}{\textbf{Total}}
& &
& \multirow{1}{*}{\textbf{Bug}} \\

\textbf{Bugs}
& \multicolumn{1}{r}{\textbf{Time (s)}}
& \multicolumn{1}{r}{\textbf{found?}}

& \multicolumn{1}{r}{\textbf{Bug (s)}}
& \multicolumn{1}{r}{\textbf{Time (s)}}
& \multicolumn{1}{r}{\textbf{\#SP}}
& \multicolumn{1}{r}{\textbf{\%Buggy}}
& \multicolumn{1}{r}{\textbf{found?}}\\[0.3em]

\toprule

vNext Liveness Bug

& \multicolumn{1}{r}{0}
& \multicolumn{1}{r}{\xmark}

& \multicolumn{1}{r}{}
& \multicolumn{1}{r}{}
& \multicolumn{1}{r}{}
& \multicolumn{1}{r}{6\%}
& \multicolumn{1}{r}{\cmark}\\

QueryAtomicFilterShadowing

& \multicolumn{1}{r}{0}
& \multicolumn{1}{r}{\xmark}

& \multicolumn{1}{r}{9.17}
& \multicolumn{1}{r}{}
& \multicolumn{1}{r}{188}
& \multicolumn{1}{r}{x\%}
& \multicolumn{1}{r}{\cmark}\\

QueryStreamedFilterShadowing

& \multicolumn{1}{r}{0}
& \multicolumn{1}{r}{\xmark}

& \multicolumn{1}{r}{}
& \multicolumn{1}{r}{}
& \multicolumn{1}{r}{}
& \multicolumn{1}{r}{x\%}
& \multicolumn{1}{r}{\cmark}\\

QueryStreamedLock

& \multicolumn{1}{r}{0}
& \multicolumn{1}{r}{\xmark}

& \multicolumn{1}{r}{}
& \multicolumn{1}{r}{}
& \multicolumn{1}{r}{}
& \multicolumn{1}{r}{x\%}
& \multicolumn{1}{r}{\cmark}\\

QueryStreamedBackUpNewStream

& \multicolumn{1}{r}{0}
& \multicolumn{1}{r}{\xmark}

& \multicolumn{1}{r}{}
& \multicolumn{1}{r}{}
& \multicolumn{1}{r}{}
& \multicolumn{1}{r}{x\%}
& \multicolumn{1}{r}{\cmark}\\

QueryStreamedSaveNewConfig

& \multicolumn{1}{r}{0}
& \multicolumn{1}{r}{\xmark}

& \multicolumn{1}{r}{}
& \multicolumn{1}{r}{}
& \multicolumn{1}{r}{}
& \multicolumn{1}{r}{x\%}
& \multicolumn{1}{r}{\cmark}\\

DeleteNoLeaveTombstonesEtag

& \multicolumn{1}{r}{0}
& \multicolumn{1}{r}{\xmark}

& \multicolumn{1}{r}{}
& \multicolumn{1}{r}{}
& \multicolumn{1}{r}{}
& \multicolumn{1}{r}{x\%}
& \multicolumn{1}{r}{\cmark}\\

DeletePrimaryKey

& \multicolumn{1}{r}{0}
& \multicolumn{1}{r}{\xmark}

& \multicolumn{1}{r}{}
& \multicolumn{1}{r}{}
& \multicolumn{1}{r}{}
& \multicolumn{1}{r}{x\%}
& \multicolumn{1}{r}{\cmark}\\

EnsurePartitionSwitchedFromPopulated

& \multicolumn{1}{r}{0}
& \multicolumn{1}{r}{\xmark}

& \multicolumn{1}{r}{}
& \multicolumn{1}{r}{}
& \multicolumn{1}{r}{}
& \multicolumn{1}{r}{x\%}
& \multicolumn{1}{r}{\cmark}\\

TombstoneOutputETag

& \multicolumn{1}{r}{0}
& \multicolumn{1}{r}{\xmark}

& \multicolumn{1}{r}{}
& \multicolumn{1}{r}{}
& \multicolumn{1}{r}{}
& \multicolumn{1}{r}{x\%}
& \multicolumn{1}{r}{\cmark}\\

MigrateSkipPreferOld

& \multicolumn{1}{r}{0}
& \multicolumn{1}{r}{\xmark}

& \multicolumn{1}{r}{}
& \multicolumn{1}{r}{}
& \multicolumn{1}{r}{}
& \multicolumn{1}{r}{x\%}
& \multicolumn{1}{r}{\cmark}\\

MigrateSkipUseNewWithTombstones

& \multicolumn{1}{r}{0}
& \multicolumn{1}{r}{\xmark}

& \multicolumn{1}{r}{}
& \multicolumn{1}{r}{}
& \multicolumn{1}{r}{}
& \multicolumn{1}{r}{x\%}
& \multicolumn{1}{r}{\cmark}\\

InsertBehindMigrator

& \multicolumn{1}{r}{0}
& \multicolumn{1}{r}{\xmark}

& \multicolumn{1}{r}{}
& \multicolumn{1}{r}{}
& \multicolumn{1}{r}{}
& \multicolumn{1}{r}{x\%}
& \multicolumn{1}{r}{\cmark}\\[0.1em]

\end{tabular}
\caption{Results from running the \psharp random and PCT systematic testing schedulers for 100,000 iterations. We report: time in seconds to find a bug (Time to Bug); number of scheduling steps when a bug was found (\#SS); and if a bug was found with a particular scheduler (BF?).}
\label{tab:testing}
\end{table*}

Table~\ref{tab:testing} presents the results from running the \psharp systematic testing engine on each case study with an enabled bug using the random and the PCT schedulers. We configured the engine to perform 100,000 iterations. The random seed for both schedulers was generated in each iteration using the \texttt{DateTime.Now.Millisecond} API which represents the current time in milliseconds. The PCT scheduler was configured with a bug depth of 2 and a max number of scheduling steps to execute of 500. All reported times are in seconds.

For MigratingTable, the upper section of the table uses the random input generator described in Section~\ref{sec:mtable:input}.  For each bug that led to at least one test failure, we manually reviewed one of the failure traces to confirm it reflected the intended bug.  For each of the remaining bugs, we repeated the test using a test case custom written to trigger that particular bug in order to confirm that the failure to detect the bug in the original test was due to unlucky random choices of inputs and schedules and not some other problem with the experimental setup.  These results are in the lower section of the table; they can serve as additional cases in which to compare the random and PCT schedulers but do not represent a testing method one could use to find unknown bugs in software.  It may have been interesting to report results for specific test cases written before we knew the bugs as mentioned in \ref{sec:mtable:input}, but we did not do so in this study.

Controlled random scheduling has proven to be efficient for finding concurrency bugs~\cite{thomson2014sct, deligiannis2015psharp}.


\section{Related Work}
\label{sec:rw}

Most related to our work are model checking~\cite{godefroid1997verisoft} and systematic concurrency testing~\cite{musuvathi2008finding, emmi2011delay, thomson2014sct}, two powerful techniques that have been widely used in the past for finding Heisenbugs in the actual implementation of distributed systems~\cite{killian2007life, yang2009modist, yabandeh2009crystalball, guerraoui2011model, guo2011practical, simsa2011dbug, gunawi2011fate, leesatapornwongsa2014samc}.

State-of-the-art model checkers, such as \textsc{MoDist}~\cite{yang2009modist} and dBug~\cite{simsa2011dbug}, typically focus on testing entire, often \emph{unmodified}, distributed systems, an approach that easily leads to state-space explosion. \textsc{DeMeter}~\cite{guo2011practical}, built on top of \textsc{MoDist}, aims to reduce the state-space when exploring unmodified distributed systems. \textsc{DeMeter} explores individual components of a large system in isolation, and then dynamically extracts interface behavior between components to perform a global exploration. In contrast, we try to offer a more pragmatic approach for handling state-space explosion. We first \emph{partially} model a distributed system using \psharp. Then, we systematically test the actual implementation of each system component against its \psharp test harness. Our approach aims to enhance unit and integration testing, techniques widely used in production, where only individual or a small number of components are tested at each time.

SAMC~\cite{leesatapornwongsa2014samc} offers a way of incorporating application-specific information during systematic testing to reduce the set of interleavings that the tool has to explore. Such techniques based on partial-order reduction~\cite{godefroid1996partial, flanagan2005dynamic} are complementary to our approach: \psharp could use them to reduce the exploration state-space. Likewise, other tools can use language technology like \psharp to write models and reduce the complexity of the system-under-test.

\textsc{MaceMc}~\cite{killian2007life} is a model checker for distributed systems written in the \textsc{Mace}~\cite{killian2007mace} language. The focus of \textsc{MaceMc} is to find liveness property violations using an algorithm based on depth-bounded random walk and the use of heuristics. Because \textsc{MaceMc} can only test systems written in \textsc{Mace}, it cannot be easily used in an industrial setting. In contrast, \psharp can be applied on legacy code written in \csharp, a mainstream language.

%\textsc{Fate} and \textsc{Destini} is a framework for systematically injecting failures in distributed systems~\cite{gunawi2011fate}. This framework focuses in exercising various failure scenarios, whereas \psharp can be used for testing generic safety and liveness properties of distributed systems.

Formal methods have been successfully used in industry to verify the correctness of distributed protocols. A recent notable example is the use of TLA+~\cite{lamport1994temporal} by the Amazon Web Services team~\cite{newcombe2015aws}. TLA+ is an expressive formal specification language that can be used to design and verify concurrent programs via model checking. A limitation of TLA+, as well as other similar specification languages, is that they are applied on a model of the system and not the actual system. Even if the model is verified, the gap between the real-world implementation and the verified model is still significant, so implementation bugs are still a realistic concern.

Another formal approach is Verdi~\cite{wilcox2015verdi}, a framework for writing and verifying distributed systems in Coq~\cite{barras1997coq}. Verdi generates OCaml code from the verifying system, which can be used for execution. In contrast, \psharp performs bounded testing on a system already written in \csharp, which in our experience lowers the bar for adoption by engineering teams. Verdi does not currently support detecting liveness property violations, an important class of bugs in distributed storage systems.


\section{Conclusion}
\label{sec:conclusion}

We presented a new methodology for testing distributed storage systems. Our approach involves using \psharp, an extension of the \csharp language that provides advanced modeling, specification and systematic testing capabilities. We reported experience on applying our methodology on three distributed storage systems used in production inside Microsoft. Using \psharp we found, reproduced and fixed numerous bugs in these case studies.

%\section{Acknowledgments}

%Draft.

% Balances the bibliography in the last page
\balance

{\footnotesize \bibliographystyle{acm}
\bibliography{references}}

\end{document}
