% This is "sig-alternate.tex" V2.0 May 2012
% This file should be compiled with V2.5 of "sig-alternate.cls" May 2012
%
% This example file demonstrates the use of the 'sig-alternate.cls'
% V2.5 LaTeX2e document class file. It is for those submitting
% articles to ACM Conference Proceedings WHO DO NOT WISH TO
% STRICTLY ADHERE TO THE SIGS (PUBS-BOARD-ENDORSED) STYLE.
% The 'sig-alternate.cls' file will produce a similar-looking,
% albeit, 'tighter' paper resulting in, invariably, fewer pages.
%
% ----------------------------------------------------------------------------------------------------------------
% This .tex file (and associated .cls V2.5) produces:
%       1) The Permission Statement
%       2) The Conference (location) Info information
%       3) The Copyright Line with ACM data
%       4) NO page numbers
%
% as against the acm_proc_article-sp.cls file which
% DOES NOT produce 1) thru' 3) above.
%
% Using 'sig-alternate.cls' you have control, however, from within
% the source .tex file, over both the CopyrightYear
% (defaulted to 200X) and the ACM Copyright Data
% (defaulted to X-XXXXX-XX-X/XX/XX).
% e.g.
% \CopyrightYear{2007} will cause 2007 to appear in the copyright line.
% \crdata{0-12345-67-8/90/12} will cause 0-12345-67-8/90/12 to appear in the copyright line.
%
% ---------------------------------------------------------------------------------------------------------------
% This .tex source is an example which *does* use
% the .bib file (from which the .bbl file % is produced).
% REMEMBER HOWEVER: After having produced the .bbl file,
% and prior to final submission, you *NEED* to 'insert'
% your .bbl file into your source .tex file so as to provide
% ONE 'self-contained' source file.
%
% ================= IF YOU HAVE QUESTIONS =======================
% Questions regarding the SIGS styles, SIGS policies and
% procedures, Conferences etc. should be sent to
% Adrienne Griscti (griscti@acm.org)
%
% Technical questions _only_ to
% Gerald Murray (murray@hq.acm.org)
% ===============================================================
%
% For tracking purposes - this is V2.0 - May 2012

\documentclass{sig-alternate}

\usepackage{xspace}

\newcommand{\psharp}{P\#\xspace}
\newcommand{\csharp}{C\#\xspace}

\begin{document}
%
% --- Author Metadata here ---
%\conferenceinfo{WOODSTOCK}{'97 El Paso, Texas USA}
%\CopyrightYear{2007} % Allows default copyright year (20XX) to be over-ridden - IF NEED BE.
%\crdata{0-12345-67-8/90/01}  % Allows default copyright data (0-89791-88-6/97/05) to be over-ridden - IF NEED BE.
% --- End of Author Metadata ---

\title{Uncovering Distributed System Bugs\\ during Testing (not Production!) -- draft title}
%
% You need the command \numberofauthors to handle the 'placement
% and alignment' of the authors beneath the title.
%
% For aesthetic reasons, we recommend 'three authors at a time'
% i.e. three 'name/affiliation blocks' be placed beneath the title.
%
% NOTE: You are NOT restricted in how many 'rows' of
% "name/affiliations" may appear. We just ask that you restrict
% the number of 'columns' to three.
%
% Because of the available 'opening page real-estate'
% we ask you to refrain from putting more than six authors
% (two rows with three columns) beneath the article title.
% More than six makes the first-page appear very cluttered indeed.
%
% Use the \alignauthor commands to handle the names
% and affiliations for an 'aesthetic maximum' of six authors.
% Add names, affiliations, addresses for
% the seventh etc. author(s) as the argument for the
% \additionalauthors command.
% These 'additional authors' will be output/set for you
% without further effort on your part as the last section in
% the body of your article BEFORE References or any Appendices.

\numberofauthors{5} %  in this sample file, there are a *total*
% of EIGHT authors. SIX appear on the 'first-page' (for formatting
% reasons) and the remaining two appear in the \additionalauthors section.
%
\author{
% You can go ahead and credit any number of authors here,
% e.g. one 'row of three' or two rows (consisting of one row of three
% and a second row of one, two or three).
%
% The command \alignauthor (no curly braces needed) should
% precede each author name, affiliation/snail-mail address and
% e-mail address. Additionally, tag each line of
% affiliation/address with \affaddr, and tag the
% e-mail address with \email.
%
% 1st. author
\alignauthor Draft\\
       \affaddr{Draft}\\
       \email{Draft}
% 2nd. author
\alignauthor Draft\\
       \affaddr{Draft}\\
       \email{Draft}
% 3rd. author
\alignauthor Draft\\
       \affaddr{Draft}\\
       \email{Draft}
\and  % use '\and' if you need 'another row' of author names
% 4th. author
\alignauthor Draft\\
       \affaddr{Draft}\\
       \email{Draft}
% 5th. author
\alignauthor Draft\\
       \affaddr{Draft}\\
       \email{Draft}
}

\date{15 August 2015}
% Just remember to make sure that the TOTAL number of authors
% is the number that will appear on the first page PLUS the
% number that will appear in the \additionalauthors section.

\maketitle
\begin{abstract}
Distributed systems are notoriously hard to test. This is due to many well-known sources of nondeterminism: races in the asynchronous interaction between system components; the use of multithreaded code inside individual components; unexpected node failures; message losses; and interaction with a complex external environment. Stress testing techniques that are commonly used in industry are unable to effectively capture and control all these sources of nondeterminism, which causes many tricky bugs to be missed during testing and only get exposed after a system has been put in production.

We present a new methodology for testing distributed systems, and uncovering bugs before these systems are released in the wild. Our approach involves the use of \psharp, an asynchronous programming language and concurrency unit testing framework, to model the environment of a distributed system, and then capture and systematically explore all sources of nondeterminism. We present two case studies of using \psharp to test production distributed systems inside Microsoft: a distributed storage management system for Windows Azure, and a live migration protocol. Using \psharp, we managed to uncover a very subtle bug that was haunting developers for a long time, as they did not have an effective way to reproduce it and nail down the culprit. \psharp uncovered the problem in a very small setting, which made it easy to examine traces and identify the problem.

\end{abstract}

\category{D.2.4}{Software Engineering}{Software/Program Verification}
\category{D.2.5}{Software Engineering}{Testing and Debugging}

\keywords{distributed systems, testing}

\section{Introduction}
\label{sec:intro}

Draft

\section{Background}
\label{sec:bg}

Draft

\section{Evaluation}
\label{sec:eval}

Draft

\section{Case studies}
\label{sec:cases}

Draft

\section{Related Work}
\label{sec:rw}

Draft

\section{Conclusions}
\label{sec:concl}

Draft.

%\section{Acknowledgments}
%Draft.

%
% The following two commands are all you need in the
% initial runs of your .tex file to
% produce the bibliography for the citations in your paper.
\bibliographystyle{abbrv}
\bibliography{sigproc}  % sigproc.bib is the name of the Bibliography in this case
% You must have a proper ".bib" file
%  and remember to run:
% latex bibtex latex latex
% to resolve all references
%
% ACM needs 'a single self-contained file'!
%
%APPENDICES are optional
%\balancecolumns

\end{document}
