% This is "sig-alternate.tex" V2.0 May 2012
% This file should be compiled with V2.5 of "sig-alternate.cls" May 2012
%
% This example file demonstrates the use of the 'sig-alternate.cls'
% V2.5 LaTeX2e document class file. It is for those submitting
% articles to ACM Conference Proceedings WHO DO NOT WISH TO
% STRICTLY ADHERE TO THE SIGS (PUBS-BOARD-ENDORSED) STYLE.
% The 'sig-alternate.cls' file will produce a similar-looking,
% albeit, 'tighter' paper resulting in, invariably, fewer pages.
%
% ----------------------------------------------------------------------------------------------------------------
% This .tex file (and associated .cls V2.5) produces:
%       1) The Permission Statement
%       2) The Conference (location) Info information
%       3) The Copyright Line with ACM data
%       4) NO page numbers
%
% as against the acm_proc_article-sp.cls file which
% DOES NOT produce 1) thru' 3) above.
%
% Using 'sig-alternate.cls' you have control, however, from within
% the source .tex file, over both the CopyrightYear
% (defaulted to 200X) and the ACM Copyright Data
% (defaulted to X-XXXXX-XX-X/XX/XX).
% e.g.
% \CopyrightYear{2007} will cause 2007 to appear in the copyright line.
% \crdata{0-12345-67-8/90/12} will cause 0-12345-67-8/90/12 to appear in the copyright line.
%
% ---------------------------------------------------------------------------------------------------------------
% This .tex source is an example which *does* use
% the .bib file (from which the .bbl file % is produced).
% REMEMBER HOWEVER: After having produced the .bbl file,
% and prior to final submission, you *NEED* to 'insert'
% your .bbl file into your source .tex file so as to provide
% ONE 'self-contained' source file.
%
% ================= IF YOU HAVE QUESTIONS =======================
% Questions regarding the SIGS styles, SIGS policies and
% procedures, Conferences etc. should be sent to
% Adrienne Griscti (griscti@acm.org)
%
% Technical questions _only_ to
% Gerald Murray (murray@hq.acm.org)
% ===============================================================
%
% For tracking purposes - this is V2.0 - May 2012

\documentclass{sig-alternate}

\usepackage{url}
\usepackage{xspace}

\newcommand{\psharp}{P\#\xspace}
\newcommand{\csharp}{C\#\xspace}

\begin{document}
%
% --- Author Metadata here ---
\conferenceinfo{WOODSTOCK}{'97 El Paso, Texas USA}
%\CopyrightYear{2007} % Allows default copyright year (20XX) to be over-ridden - IF NEED BE.
%\crdata{0-12345-67-8/90/01}  % Allows default copyright data (0-89791-88-6/97/05) to be over-ridden - IF NEED BE.
% --- End of Author Metadata ---

\title{Uncovering Distributed System Bugs\\ during Testing (not Production!) -- draft title}
%
% You need the command \numberofauthors to handle the 'placement
% and alignment' of the authors beneath the title.
%
% For aesthetic reasons, we recommend 'three authors at a time'
% i.e. three 'name/affiliation blocks' be placed beneath the title.
%
% NOTE: You are NOT restricted in how many 'rows' of
% "name/affiliations" may appear. We just ask that you restrict
% the number of 'columns' to three.
%
% Because of the available 'opening page real-estate'
% we ask you to refrain from putting more than six authors
% (two rows with three columns) beneath the article title.
% More than six makes the first-page appear very cluttered indeed.
%
% Use the \alignauthor commands to handle the names
% and affiliations for an 'aesthetic maximum' of six authors.
% Add names, affiliations, addresses for
% the seventh etc. author(s) as the argument for the
% \additionalauthors command.
% These 'additional authors' will be output/set for you
% without further effort on your part as the last section in
% the body of your article BEFORE References or any Appendices.

\numberofauthors{7} %  in this sample file, there are a *total*
% of EIGHT authors. SIX appear on the 'first-page' (for formatting
% reasons) and the remaining two appear in the \additionalauthors section.
%
\author{
% You can go ahead and credit any number of authors here,
% e.g. one 'row of three' or two rows (consisting of one row of three
% and a second row of one, two or three).
%
% The command \alignauthor (no curly braces needed) should
% precede each author name, affiliation/snail-mail address and
% e-mail address. Additionally, tag each line of
% affiliation/address with \affaddr, and tag the
% e-mail address with \email.
%
% 1st. author
\alignauthor Nikolaj Bjorner\\
       \affaddr{Microsoft Research}\\
       \affaddr{Redmond, WA, USA}\\
       \email{nbjorner@microsoft.com}
% 2nd. author
\alignauthor Pantazis Deligiannis\\
       \affaddr{Imperial College London}\\
       \affaddr{London, UK}\\
       \email{p.deligiannis@imperial.ac.uk}
% 3th. author
\alignauthor John Erickson\\
       \affaddr{Microsoft}\\
       \affaddr{Redmond, WA, USA}\\
       \email{jerick@microsoft.com}
% 4th. author
\alignauthor Cheng Huang\\
       \affaddr{Microsoft Research}\\
       \affaddr{Redmond, WA, USA}\\
       \email{cheng.huang@microsoft.com}
\and  % use '\and' if you need 'another row' of author names
% 5rd. author
\alignauthor Matt McCutchen\\
       \affaddr{MIT}\\
       \affaddr{Cambridge, MA, USA}\\
       \email{rmccutch@mit.edu}
% 6th. author
\alignauthor Wolfram Schulte\\
       \affaddr{Microsoft}\\
       \affaddr{Redmond, WA, USA}\\
       \email{schulte@microsoft.com}
% 7th. author
\alignauthor Shaz Qadeer\\
       \affaddr{Microsoft Research}\\
       \affaddr{Redmond, WA, USA}\\
       \email{qadeer@microsoft.com}
}

\date{15 August 2015}
% Just remember to make sure that the TOTAL number of authors
% is the number that will appear on the first page PLUS the
% number that will appear in the \additionalauthors section.

\maketitle
\begin{abstract}
Distributed systems are notoriously hard to test. This is due to many well-known sources of nondeterminism: races in the asynchronous interaction between system components; the use of multithreaded code inside individual components; unexpected node failures; message losses; and interaction with a complex external environment. Stress testing techniques, commonly used in industry today, are unable to capture and control all these sources of nondeterminism. This results in many tricky bugs be missed during testing and only get exposed after a system has been put in production.

We present a new methodology for programming, modeling and testing distributed systems, and uncovering bugs before these systems are released in the wild. Our approach involves the use of \psharp, an asynchronous programming language and concurrency unit testing framework, which provides capabilities for modeling the environment of a distributed system, and then captures and systematically explores all sources of nondeterminism. We present two case studies of using \psharp to test production distributed systems inside Microsoft: a distributed storage management system for Windows Azure, and a live migration protocol. Using \psharp, we managed to uncover a very subtle bug that was haunting developers for a long time, as they did not have an effective way to reproduce it and nail down the culprit. \psharp uncovered the problem in a very small setting, which made it easy to examine traces and identify the problem.

\end{abstract}

\category{D.2.4}{Software Engineering}{Software/Program Verification}
\category{D.2.5}{Software Engineering}{Testing and Debugging}

\keywords{distributed systems, testing}

\section{Introduction}
\label{sec:intro}

Distributed systems are notoriously hard to test. This is due to many well-known sources of \emph{nondeterminism}, such as race conditions in the asynchronous interaction between system components, the use of multithreaded code inside a component, unexpected node failures, unreliable communication channels and message losses, and interaction with clients and external services.
%
All these sources of nondeterminism translate into \emph{exponentially} many execution paths that a distributed system might potentially execute. A bug might hide deep inside one of these paths and only manifest under extremely rare corner cases.

Classic techniques that product groups employ to test, verify and validate their systems (such as code reviews, stress testing, fault injection, and chaos testing) are unable to capture and control all the aforementioned sources of nondeterminism, which causes the most tricky backs being missed during testing and only getting exposed after a system has been put in production. Discovering and fixing bugs in production, though, is bad for business as it can cost a lot of money and many dissatisfied customers.

We interviewed engineers from the Microsoft Azure team regarding the top problems in distributed system development, and the unified response was that the most critical problem today is how to improve \emph{testing coverage} to find bugs \emph{before} a system goes in production. The need for better testing techniques is not specific to Microsoft; other companies, such as Amazon and Google, publicly acknowledge~\cite{newcombe2015aws} that testing methodologies have to improve to be able to reason about the correctness of increasingly more complex distributed systems.

Amazon recently published an article~\cite{newcombe2015aws} that describes their use of TLA+~\cite{lamport1994temporal} to detect distributed system bugs and prevent them from reaching production. TLA+ is a powerful specification language for verifying distributed protocols, but it is unable to verify the code that is actually being executed. The implied assumption is that a model of the system will be verified, and then the programmers are responsible to match what was verified with the source code of the real system. Although many design bugs can be caught with this approach, there is \emph{no guarantee} that the real distributed system will be free of bugs.

In this work, our goal is to \emph{test what is being executed}. We present a new methodology for testing legacy distributed systems and uncovering bugs before these systems are released in the wild. We achieve this using \psharp~\cite{deligiannis2015psharp}, an extension of the mainstream language \csharp that provides two key capabilities: (i) a flexible way of modeling the environment using simple language features; and (ii) a systematic concurrency testing framework that is able to capture and take control of all the nondeterminism in a real system (together with its modeled environment) and systematically explore execution paths to discover bugs.

We present two case studies of using \psharp to test production distributed systems for Windows Azure inside Microsoft: a distributed storage management system and a live migration protocol. Using \psharp, we managed to uncover a very subtle bug that was haunting developers for a long time as they did not have an effective way to reproduce the bug and nail down the culprit. \psharp uncovered this bug in a very small setting, which made it easy to examine traces, identify and eventually fix the problem.

To summarize, our contributions are as follows:

\begin{itemize}
\item We present a methodology that allows flexible modeling of the environment of a distributed system using simple language mechanisms.
\item Our infrastructure can test production code written in \csharp, which is a mainstream language.
\item We present two case studies of using \psharp to test production distributed systems, finding bugs that could not be found with traditional testing techniques.
\item ...
\end{itemize}


\section{Overview}
\label{sec:overview}

How cheng was testing his code before and how after

The goal of our work is to develop testing techniques for detecting and debugging Heisenbugs in distributed storage systems prior to deployment. To achieve our goal, we use \psharp~\cite{deligiannis2015psharp}, an extension of the mainstream \csharp language that provides: (i) language support for \emph{modeling the environment} of distributed systems developed using Microsoft's .NET framework; and (ii) a \emph{testing engine} that can systematically explore all interleavings between asynchronous event handlers, as well as other nondeterministic events such as failures and timeouts.

A \psharp program consists of multiple \emph{state machines} that communicate with each other \emph{asynchronously} by sending and receiving \emph{events}, which may contain an optional \emph{payload}. A \psharp machine declaration is similar to a class declaration in \csharp: it can contain an arbitrary number of fields and methods. However, a machine declaration differs from a class declaration, as it also contains an input event queue, and one or more \emph{states}. Each state can register \emph{actions} to handle incoming events.

\psharp machines run concurrently with each other, each executing an event handling loop that dequeues an event from the input queue and handles it by invoking the registered action. This action might access a field, call a method, transition the machine to a new state, create a new machine, or send an event to another machine. In \psharp, a send operation is \emph{non-blocking}; the event is simply enqueued into the input queue of the target machine, which will dequeue and handle the event concurrently. All this functionality is provided in a lightweight runtime library, built on top of Microsoft's Task Parallel Library~\cite{leijen2009tpl}.

Our approach of using \psharp to test distributed systems, implemented in .NET, requires the developer to perform the following three key modeling tasks. These tasks are illustrated in \S\ref{sec:method} for the Azure Storage vNext system.

\begin{enumerate}
\item
The computation model underlying \psharp is communicating state machines. The environment of the system-under-test must be modeled using the \psharp APIs, while the real components of the system must be wrapped inside \psharp machines. We call this modeled environment the \psharp test harness.

\item
All the asynchrony due to message passing between system components must become explicit and be modeled using the \psharp event-sending APIs. Similarly, all other sources of nondeterminism (e.g. failures and timer expiration) must be modeled using appropriate \psharp APIs. This step allows the \psharp runtime to systematically explore all asynchronous event handler interleavings and all other sources of nondeterminism during testing.

\item
The criteria for correctness of an execution must be specified. Specifications in \psharp can encode either \emph{safety} or \emph{liveness}~\cite{lamport1977proving} properties. Safety specifications generalize the notion of source code assertions; a safety violation is a finite trace leading to an erroneous state. Liveness specifications generalize nontermination; a liveness violation is an infinite trace that exhibits lack of progress.
\end{enumerate}

\noindent
\psharp uses modern object-oriented language features such as \emph{interfaces} and \emph{virtual method dispatch} to connect the real code with the modeled code. Programmers in industry are used to working with such features, and heavily use them in production for testing purposes. In our experience, this significantly lowers the bar for product teams inside Microsoft to embrace \psharp for testing.

In principle, our modeling methodology is \emph{not specific} to \psharp and the .NET framework, and can be used in combination with any other programming framework that has equivalent capabilities. We also argue that our approach is \emph{flexible} since it allows the user to model \emph{as much} or \emph{as little} of the environment as required to achieve the desired level of testing.

\subsection{Specifications}
\label{sec:bg:bugs}

In  this section, we describe how safety and liveness specifications are expressed in \psharp.

\textbf{Safety specifications}.
In addition to the usual assertions for stating safety properties that are local to a machine, \psharp also provides a way to specify global assertions by using a \emph{safety monitor}~\cite{desai2015building}, which is a special machine that can receive, but not send, events.

A safety monitor maintains local state that is modified in response to events received from ordinary (non-monitor) machines. This local state is used to maintain a history of the computation that is relevant to the property being specified. An erroneous global behavior is flagged via an assertion on the private state of the safety monitor. Thus, a \psharp monitor cleanly separates instrumentation state required for specification (inside the monitor) from program state (outside the monitor).

\textbf{Liveness specifications}.
Liveness property specifications are more difficult to encode since they are intended to capture program progress over infinite executions. Violation of a liveness property can only be demonstrated by an infinite execution. Usually, a liveness property is specified via a temporal logic formula~\cite{Pnueli1977,lamport1994temporal}. We take a different approach and allow the programmer to write a \emph{liveness monitor}~\cite{desai2015building}. Similar to a safety monitor, a liveness monitor can receive, but not send, events. Unlike a safety monitor, a liveness monitor contains two types of states: \emph{hot} and \emph{cold}.

A hot state denotes a point in the execution where progress is required but has not happened yet; for example, a node has failed but a new one has not come up yet. A liveness monitor enters a hot state upon receiving notification of an event that requires the system to make progress. The liveness monitor leaves the hot state and enters a cold state upon receiving another event notifying it of progress. An infinite execution is erroneous if the liveness monitor is in a hot state infinitely often but in a cold state only finitely often. Our liveness monitors can encode arbitrary temporal logic properties. A thermometer analogy is useful for understanding liveness monitors: a hot state increases the temperature by a small value; a cold state resets the temperature to zero; a liveness error happens if the temperature becomes infinite.
%The distinction between ordinary and cold states is useful for modeling progress for each instance of an infinite stream of events, such as a periodic request or a periodic timer expiration.

A liveness violation is witnessed by an \emph{infinite} execution in which all concurrently executing \psharp machines are \emph{fairly} scheduled. Unfortunately, it is clearly not possible to generate an infinite execution by executing a program for a finite amount of time. Therefore, our implementation of liveness checking in \psharp approximates an infinite execution using several heuristics. One such heuristic considers an execution longer than a large user-supplied bound as an ``infinite'' execution~\cite{killian2007life, musuvathi2008fair}. Another heuristic maintains a cache of (pieces of) the state of the \psharp program obtained at each step in the execution, and reports an ``infinite'' execution when the latest step results in a cache hit, thus approximating a cycle in the state graph of the program.

%\textbf{Using monitors}.
%To declare a safety or a liveness monitor, the programmer must inherit from the \psharp \texttt{Monitor} class. \psharp monitors are singleton instances and, thus, an ordinary machine does not need a reference to a monitor to send it an event. Instead, an ordinary machine can invoke a monitor by calling \texttt{Monitor<M>(e, p)}, where the parameter \texttt{e} is the event being send, and the parameter \texttt{p} is an optional payload.

\subsection{Systematic testing}
\label{sec:psharp:testing}

The \psharp runtime is a lightweight layer build on top of the Task Parallel Library (TPL) of .NET that implements the semantics of \psharp: creating state machines, executing them concurrently using the default task scheduler of TPL, and sending events and enqueueing them in the appropriate machines. A key capability of the \psharp runtime is that it can execute in \emph{bug-finding} mode, which systematically tests a \psharp program to find bugs using the safety and liveness monitors that were discussed in \S\ref{sec:bg:bugs}.

When the \psharp runtime executes in bug-finding mode, an embedded \emph{systematic concurrency testing engine}~\cite{godefroid1997verisoft, musuvathi2008finding, emmi2011delay} captures and takes control of all sources of non-determinism that are \emph{known} to the \psharp runtime. In particular, the runtime is aware of nondeterminism due to the interleaving of event handlers in different machines. Each send operation to an ordinary (non-monitor) machine, and each create operation of an ordinary machine, creates an interleaving point where the runtime makes a scheduling decision. The runtime is also aware of all nondeterministic events (e.g. failures) that were captured during the modeling process.

The \psharp testing engine will repeatedly execute a program from start to completion, each time exploring a potentially different set of scheduling decisions, until it either reaches a bound (in number of iterations or time), or it hits a property violation. This testing process is fully automatic, has no false-positives (assuming an accurate environment model), and can reproduce found bugs by replaying buggy schedules. After a bug is found, the engine generates a trace that represents the buggy schedule. In contrast to logs typically generated during production, the \psharp trace provides a global order of all communication events and, thus, is easier to debug.

We have implemented two different schedulers inside the \psharp systematic testing engine: a \emph{random} scheduler that makes each nondeterministic choice randomly, and a \emph{probabilistic concurrency testing}~\cite{burckhardt2010pct} (PCT) scheduler. It is straightforward to create a new scheduler by implementing the \texttt{ISchedulingStrategy} interface~\cite{desai2015systematic} exposed by the \psharp libraries. The interface exposes callbacks that are invoked by the \psharp testing engine for taking decisions regarding which machine to schedule next, and can be used for developing both generic and application-specific schedulers.

\textbf{Handling \csharp 5.0 asynchrony}.
The Live Azure Table Migration system (see \S\ref{sec:cases:migration}) heavily uses the \texttt{async} and \texttt{await} \csharp 5.0 language primitives. An \texttt{async} method is able to perform \texttt{await} on a TPL task \emph{without} blocking the current thread. The \csharp compiler achieves this non-blocking behavior by wrapping the code following an \texttt{await} statement in a TPL \emph{task continuation}, and then returning execution to the caller of the \texttt{async} method. The TPL scheduler executes this task continuation when the task being awaited has completed. In principle, this asynchrony can be modeled using the \psharp primitives for machine creation and message passing. However, in our experience, modeling async/await asynchronous code using \psharp is difficult and time consuming, and thus infeasible for large code bases. To tackle this problem, we developed a solution that automatically captures asynchronously created TPL tasks and wraps them in temporary \psharp machines.

Our solution involves using a \emph{custom task scheduler}, during systematic testing with \psharp, instead of the default TPL scheduler. Assuming that the developer does not explicitly change the task scheduler (something extremely uncommon), any TPL tasks spawned in the \psharp program will be scheduled for execution in the \psharp task scheduler. These tasks will be intercepted by our custom task scheduler and wrapped inside a special, available only internally, task-machine. When the \psharp testing engine schedules a task-machine for execution, the corresponding task is unwrapped and executed. This approach works nicely with the \texttt{async} and \texttt{await} \csharp 5.0 primitives: when an \texttt{async} method is called, the asynchronous TPL task will be automatically wrapped in a task-machine and scheduled by \psharp; for \texttt{await}, no special treatment is required since the task continuation will also be automatically wrapped and scheduled accordingly.


\section{Background}
\label{sec:bg}

Distributed systems typically consist of two or more components that communicate \emph{asynchronously} by sending and receiving messages through a network layer~\cite{lamport1978time}. Each component has its own input message queue, and when a message arrives, the component responds by executing an appropriate \emph{message handler}. Such a handler consists of a sequence of program statements that might update the internal state of the component, send a message to another component in the system, or even create an entirely new component.

\subsection{Challenges in testing}
\label{sec:bg:challenges}

In a distributed system, message handlers can interleave in arbitrary order, because of the asynchronous nature of message-based communication. To complicate matters further, unexpected failures are the norm in production systems: nodes in a cluster might fail at any moment, and thus programmers have to implement sophisticated mechanisms that can deal with these failures and recover the state of the system. Moreover, with multicore machines having become a commodity, individual components of a distributed system are commonly implemented using multithreaded code, which adds another source of nondeterminism.

All the above sources of nondeterminism (as well as nondeterminism due to timeouts, message losses and client requests) can easily create \emph{heisenbugs}~\cite{gray1986computers, musuvathi2008finding}, which are corner-case bugs that are difficult to detect, diagnose and fix, without using advanced \emph{asynchrony-aware} testing techniques. Techniques such as unit testing, integration testing and stress testing are heavily used in industry today for finding bugs in production code. However, these techniques are not effective for testing distributed systems, as they are not able to capture and control the many sources of nondeterminism.

The ideal testing technique should be able to work on unmodified distributed systems, capture and control all possible sources of nondeterminism, systematically inject faults in the right places, and explore all feasible execution paths. However, this is easier said than done when testing production systems.

A completely different approach for reasoning about the correctness of distributed systems is to use formal methods.  A notable example is TLA+~\cite{lamport1994temporal}, a formal specification language that can be used to design and verify concurrent programs via model checking. Amazon recently published an article describing their use of TLA+ in Amazon Web Services to verify distributed protocols~\cite{newcombe2015aws}. A limitation of TLA+, as well as other similar specification languages, is that they are applied on a model of the system and not the actual system. Even if the model is verified, the gap between a real-world implementation and the verified model is still significant, so implementation bugs are still a realistic concern.

\subsection{Types of bugs}
\label{sec:bg:bugs}

We can classify most distributed system bugs in two categories: \emph{safety} and \emph{liveness} property violations~\cite{lamport1977proving}.

\begin{description}
\item[Safety] A safety property checks that an erroneous program state is \emph{never} reached, and is satisfied if it \emph{always} holds in each possible program execution.

\item[Liveness] A liveness property checks that some progress \emph{will} happen, and is satisfied if it \emph{always eventually} holds in each possible program execution.
\end{description}

\noindent
A safety property can be specified using an \emph{assertion} that fails if the property gets violated in some program state. An example of a generic safety property for message passing systems is to assert that whenever a message gets dequeued there must be an action that can handle the received message.

Liveness properties are much harder to specify and check since they apply over entire program executions and not just individual program states. Normally, liveness checking requires the identification of an infinite fair execution that never satisfies the liveness property~\cite{schuppan2004efficient, musuvathi2008fair}. Prior work~\cite{schuppan2004efficient} has proposed that assuming a program with finite state space, a liveness property can be converted into a safety property. Other researchers proposed the use of heuristics and only exploring finite executions of an infinite state space system using random walks to identify if a liveness property is violated~\cite{killian2007life}.


Dependency injection pattern

\section{Case studies}
\label{sec:cases}

\subsection{Distributed Extent Management System}

We used \psharp to test the \emph{distributed extent management} component of the Windows Azure vNext distributed storage system. This component is responsible for managing the partitioned extent metadata and works as follows.

There are multiple \emph{extent managers} (EMgrs) that are in charge of managing a subset of the extents. Each EMgr communicates asynchronously (via remote procedure calls) with a number of \emph{extent nodes} (ENs) that store the extents. Each EN sends: (i) a \emph{heartbeat} every 5 seconds, to notify the EMgr that it is still available; and (ii) a \emph{sync report} every 5 minutes, to synchronize its extent with the rest of the nodes. The EMgr contains two data structures: an \emph{extent center} (ECtr), which is updated every time an EN syncs; and an EN map, which is updated every time it receives a heartbeat from an EN. The EN map runs an EN expiration loop that is responsible for removing ENs that have expired from the EN map and also delete them from the ECtr. Finally, the EMgr runs an extent repair loop, which examines all contents of all extents in the ECtr and schedules repairs of extents if they are out-of-date (via a repair request).

\section{Evaluation}
\label{sec:eval}

Draft

\section{Related Work}
\label{sec:rw}

\psharp is most closely related to \textsc{MoDist}~\cite{yang2009modist}, a model checker that can be applied on unmodified distributed systems. To achieve this, the tool uses binary instrumentation (reuses the instrumentation engine of the D$^3$S testing tool) to expose all actions of the system under test, and then uses a model checking engine to systematically explore these actions. \textsc{MoDist} also provides a virtual clock manager that can simulate timeouts and accelerate the passage of time. A major limitation of \textsc{MoDist} is that it must be applied on a whole system, which does not scale for production systems. In contrast, \psharp has built-in support for modeling the environment of individual components of a system, which allows the tool to achieve scalability. Further, \psharp is an extension of the \csharp language and has built-in systematic testing support; it does not need to perform binary instrumentation which can be computationally expensive.

\textsc{MaceMc}~\cite{killian2007life} is a model checker for distributed systems implemented in the \textsc{Mace} language. The focus of \textsc{MaceMc} is to find liveness property violations using an algorithm based on bounded random walk and the use of heuristics. \psharp differs from \textsc{MaceMc} in that it can be applied on legacy code written in a mainstream language, whereas a system to be tested with \textsc{MaceMc} has to be written in \textsc{Mace}, which makes \textsc{MaceMc} harder to be applied in an industrial setting. Further, \psharp uses a simpler, but effective, algorithm for detecting liveness bugs.

\textsc{Fate} and \textsc{Destini} is a framework for systematically injecting (combination of) failures in distributed systems~\cite{gunawi2011fate}. This framework focuses in exercising failure scenarios, whereas \psharp can be used for testing generic safety and liveness properties of distributed systems. \psharp also provides language extensions and modeling capabilities that further differentiate it from prior work.

DieCast~\cite{gupta2008diecast}

A completely different approach for reasoning about the correctness of distributed systems is to use formal methods.  A notable example is TLA+~\cite{lamport1994temporal}, a formal specification language that can be used to design and verify concurrent programs via model checking. Amazon recently published an article describing their use of TLA+ in Amazon Web Services to verify distributed protocols~\cite{newcombe2015aws}. A limitation of TLA+, as well as other similar specification languages, is that they are applied on a model of the system and not the actual system. Even if the model is verified, the gap between a real-world implementation and the verified model is still significant, so implementation bugs are still a realistic concern.

Another approach is Verdi~\cite{wilcox2015verdi}, where a distributed system is written and verified in Coq, and then OCaml code is produced for execution. Verdi cannot find liveness bugs. \psharp is also more production-friendly: it works on \csharp, which is a mainstream language.


\section{Conclusion}
\label{sec:concl}

Draft.

%\section{Acknowledgments}
%Draft.

%
% The following two commands are all you need in the
% initial runs of your .tex file to
% produce the bibliography for the citations in your paper.
\bibliographystyle{abbrv}
\bibliography{references}  % sigproc.bib is the name of the Bibliography in this case
% You must have a proper ".bib" file
%  and remember to run:
% latex bibtex latex latex
% to resolve all references
%
% ACM needs 'a single self-contained file'!
%
%APPENDICES are optional
%\balancecolumns

\end{document}
