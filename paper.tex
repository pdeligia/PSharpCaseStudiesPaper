% This is "sig-alternate.tex" V2.0 May 2012
% This file should be compiled with V2.5 of "sig-alternate.cls" May 2012
%
% This example file demonstrates the use of the 'sig-alternate.cls'
% V2.5 LaTeX2e document class file. It is for those submitting
% articles to ACM Conference Proceedings WHO DO NOT WISH TO
% STRICTLY ADHERE TO THE SIGS (PUBS-BOARD-ENDORSED) STYLE.
% The 'sig-alternate.cls' file will produce a similar-looking,
% albeit, 'tighter' paper resulting in, invariably, fewer pages.
%
% ----------------------------------------------------------------------------------------------------------------
% This .tex file (and associated .cls V2.5) produces:
%       1) The Permission Statement
%       2) The Conference (location) Info information
%       3) The Copyright Line with ACM data
%       4) NO page numbers
%
% as against the acm_proc_article-sp.cls file which
% DOES NOT produce 1) thru' 3) above.
%
% Using 'sig-alternate.cls' you have control, however, from within
% the source .tex file, over both the CopyrightYear
% (defaulted to 200X) and the ACM Copyright Data
% (defaulted to X-XXXXX-XX-X/XX/XX).
% e.g.
% \CopyrightYear{2007} will cause 2007 to appear in the copyright line.
% \crdata{0-12345-67-8/90/12} will cause 0-12345-67-8/90/12 to appear in the copyright line.
%
% ---------------------------------------------------------------------------------------------------------------
% This .tex source is an example which *does* use
% the .bib file (from which the .bbl file % is produced).
% REMEMBER HOWEVER: After having produced the .bbl file,
% and prior to final submission, you *NEED* to 'insert'
% your .bbl file into your source .tex file so as to provide
% ONE 'self-contained' source file.
%
% ================= IF YOU HAVE QUESTIONS =======================
% Questions regarding the SIGS styles, SIGS policies and
% procedures, Conferences etc. should be sent to
% Adrienne Griscti (griscti@acm.org)
%
% Technical questions _only_ to
% Gerald Murray (murray@hq.acm.org)
% ===============================================================
%
% For tracking purposes - this is V2.0 - May 2012

\documentclass{sig-alternate}

\usepackage{url}
\usepackage{xspace}

\newcommand{\psharp}{P\#\xspace}
\newcommand{\csharp}{C\#\xspace}

\usepackage{color}
\newcommand{\PDComment}[1]{\textcolor{cyan}{Pantazis: #1}}

\begin{document}
%
% --- Author Metadata here ---
\conferenceinfo{WOODSTOCK}{'97 El Paso, Texas USA}
%\CopyrightYear{2007} % Allows default copyright year (20XX) to be over-ridden - IF NEED BE.
%\crdata{0-12345-67-8/90/01}  % Allows default copyright data (0-89791-88-6/97/05) to be over-ridden - IF NEED BE.
% --- End of Author Metadata ---

\title{Uncovering Distributed System Bugs\\ during Testing (not Production!)}
%
% You need the command \numberofauthors to handle the 'placement
% and alignment' of the authors beneath the title.
%
% For aesthetic reasons, we recommend 'three authors at a time'
% i.e. three 'name/affiliation blocks' be placed beneath the title.
%
% NOTE: You are NOT restricted in how many 'rows' of
% "name/affiliations" may appear. We just ask that you restrict
% the number of 'columns' to three.
%
% Because of the available 'opening page real-estate'
% we ask you to refrain from putting more than six authors
% (two rows with three columns) beneath the article title.
% More than six makes the first-page appear very cluttered indeed.
%
% Use the \alignauthor commands to handle the names
% and affiliations for an 'aesthetic maximum' of six authors.
% Add names, affiliations, addresses for
% the seventh etc. author(s) as the argument for the
% \additionalauthors command.
% These 'additional authors' will be output/set for you
% without further effort on your part as the last section in
% the body of your article BEFORE References or any Appendices.

\numberofauthors{6} %  in this sample file, there are a *total*
% of EIGHT authors. SIX appear on the 'first-page' (for formatting
% reasons) and the remaining two appear in the \additionalauthors section.
%
\author{
% You can go ahead and credit any number of authors here,
% e.g. one 'row of three' or two rows (consisting of one row of three
% and a second row of one, two or three).
%
% The command \alignauthor (no curly braces needed) should
% precede each author name, affiliation/snail-mail address and
% e-mail address. Additionally, tag each line of
% affiliation/address with \affaddr, and tag the
% e-mail address with \email.
%
% 1st. author
\alignauthor Pantazis Deligiannis\\
       \affaddr{Imperial College London}\\
       \affaddr{London, UK}\\
       \email{p.deligiannis@imperial.ac.uk}
% 2nd. author
\alignauthor John Erickson\\
       \affaddr{Microsoft}\\
       \affaddr{Redmond, WA, USA}\\
       \email{jerick@microsoft.com}
% 3rd. author
\alignauthor Cheng Huang\\
       \affaddr{Microsoft Research}\\
       \affaddr{Redmond, WA, USA}\\
       \email{cheng.huang@microsoft.com}
\and  % use '\and' if you need 'another row' of author names
% 4th. author
\alignauthor Matt McCutchen\\
       \affaddr{MIT}\\
       \affaddr{Cambridge, MA, USA}\\
       \email{rmccutch@mit.edu}
% 5th. author
\alignauthor Shaz Qadeer\\
       \affaddr{Microsoft Research}\\
       \affaddr{Redmond, WA, USA}\\
       \email{qadeer@microsoft.com}
% 6th. author
\alignauthor Wolfram Schulte\\
       \affaddr{Microsoft}\\
       \affaddr{Redmond, WA, USA}\\
       \email{schulte@microsoft.com}
}

\date{17 August 2015}
% Just remember to make sure that the TOTAL number of authors
% is the number that will appear on the first page PLUS the
% number that will appear in the \additionalauthors section.

\maketitle
\begin{abstract}
Testing distributed systems is very challenging due to multiple sources of nondeterminism, such as arbitrary interleavings between event handlers and unexpected node failures. Stress testing, commonly used in industry today, is unable to deal with this kind of nondeterminism, which results in the most tricky bugs being missed during testing and only getting exposed in production.

We present a new methodology for systematically testing unmodified distributed systems. Our approach involves the use of \psharp, an extension of \csharp that combines a flexible environmental modeling approach with a concurrency testing framework, which can capture and systematically explore all sources of nondeterminism. We present two case studies of using \psharp to test production distributed systems inside Microsoft. Using \psharp, we managed to uncover a very subtle bug that was haunting developers for a long time, as they did not have an effective way to reproduce it. \psharp uncovered the bug in a very small setting, which made it easy to examine traces, identify and fix the problem.

\end{abstract}

\category{D.2.4}{Software Engineering}{Software/Program Verification}
\category{D.2.5}{Software Engineering}{Testing and Debugging}

\keywords{distributed systems, testing}

\section{Introduction}
\label{sec:intro}

Distributed systems are notoriously hard to design, implement and test~\cite{cavage2013there, laguna2015debugging, maddox2015test}. This is due to many well-known sources of \emph{nondeterminism}~\cite{chandra2007paxos}, such as race conditions in the asynchronous interaction between system components, the use of multithreaded code inside a component, unexpected node failures, unreliable communication channels and data losses, and interaction with (human) clients.
%
All these sources of nondeterminism translate into \emph{exponentially} many execution paths that a distributed system might potentially execute. A bug might hide deep inside one of these paths and only manifest under extremely rare corner cases~\cite{gray1986computers, musuvathi2008finding}.

Classic techniques that product groups employ to test, verify and validate their systems (such as code reviews, unit testing, stress testing and fault injection) are unable to capture and control all the aforementioned sources of nondeterminism, which causes the most tricky bugs being missed during testing and only getting exposed after a system has been put in production. Discovering and fixing bugs in production, though, is bad for business as it can cost a lot of money and many dissatisfied customers.

We interviewed engineers from the Microsoft Azure team regarding the top problems in distributed system development, and the unified response was that the most critical problem today is how to improve \emph{testing coverage} to find bugs \emph{before} a system goes in production. The need for better testing techniques is not specific to Microsoft; other companies, such as Amazon and Google, publicly acknowledge~\cite{newcombe2015aws} that testing methodologies have to improve to be able to reason about the correctness of increasingly more complex distributed systems.

Amazon recently published an article~\cite{newcombe2015aws} that describes their use of TLA+~\cite{lamport1994temporal} to detect distributed system bugs and prevent them from reaching production. TLA+ is a powerful specification language for verifying distributed protocols, but it is unable to verify the code that is actually being executed. The implied assumption is that a model of the system will be verified, and then the programmers are responsible to match what was verified with the source code of the real system. Although many design bugs can be caught with this approach, there is \emph{no guarantee} that the real distributed system will be free of bugs.

In this work, our goal is to \emph{test what is being executed}. We present a new methodology for testing legacy distributed systems and uncovering bugs before these systems are released in the wild. We achieve this using \psharp~\cite{deligiannis2015psharp}, an extension of the mainstream language \csharp that provides two key capabilities: (i) a flexible way of modeling the environment using simple language features; and (ii) a systematic concurrency testing framework that is able to capture and take control of all the nondeterminism in a real system (together with its modeled environment) and systematically explore execution paths to discover bugs.

We present three case studies of using \psharp to test production distributed systems for Windows Azure inside Microsoft: a distributed storage management system and a live migration protocol. Using \psharp, we managed to uncover a very subtle bug that was haunting developers for a long time as they did not have an effective way to reproduce the bug and nail down the culprit. \psharp uncovered this bug in a very small setting, which made it easy to examine traces, identify and eventually fix the problem.

To summarize, our contributions are as follows:

\begin{itemize}
\item We present a methodology that allows flexible modeling of the environment of a distributed system using simple language mechanisms.
\item Our infrastructure can test production code written in \csharp, which is a mainstream language.
\item We present three case studies of using \psharp to test production distributed systems, finding bugs that could not be found with traditional testing techniques.
\item ...
\end{itemize}


\section{Overview}
\label{sec:overview}

Distributed systems typically consist of two or more components that communicate \emph{asynchronously} by sending and receiving messages through a network layer~\cite{lamport1978time}. Each component has its own input message queue, and when a message arrives, the component responds by executing an appropriate \emph{message handler}. Such a handler consists of a sequence of program statements that might update the internal state of the component, send a message to another component in the system, or even create an entirely new component.

\subsection{Challenges in testing}
\label{sec:bg:challenges}

In a distributed system, message handlers can interleave in arbitrary order, because of the asynchronous nature of message-based communication. To complicate matters further, unexpected failures are the norm in production systems: nodes in a cluster might fail at any moment, and thus programmers have to implement sophisticated mechanisms that can deal with these failures and recover the state of the system. Moreover, with multicore machines having become a commodity, individual components of a distributed system are commonly implemented using multithreaded code, which adds another source of nondeterminism.

All the above sources of nondeterminism (as well as nondeterminism due to timeouts, message losses and client requests) can easily create \emph{heisenbugs}~\cite{gray1986computers, musuvathi2008finding}, which are corner-case bugs that are difficult to detect, diagnose and fix, without using advanced \emph{asynchrony-aware} testing techniques. Techniques such as unit testing, integration testing and stress testing are heavily used in industry today for finding bugs in production code. However, these techniques are not effective for testing distributed systems, as they are not able to capture and control the many sources of nondeterminism.

The ideal testing technique should be able to work on unmodified distributed systems, capture and control all possible sources of nondeterminism, systematically inject faults in the right places, and explore all feasible execution paths. However, this is easier said than done when testing production systems.

A completely different approach for reasoning about the correctness of distributed systems is to use formal methods.  A notable example is TLA+~\cite{lamport1994temporal}, a formal specification language that can be used to design and verify concurrent programs via model checking. Amazon recently published an article describing their use of TLA+ in Amazon Web Services to verify distributed protocols~\cite{newcombe2015aws}. A limitation of TLA+, as well as other similar specification languages, is that they are applied on a model of the system and not the actual system. Even if the model is verified, there is no guarantee that the code that will actually execute is free of bugs. \SCComment{replace the last sentence with "the gap between a real-world implementation and the verified model is still significant, so implementation bugs are still a realistic concern." ?}

\subsection{Types of bugs}
\label{sec:bg:bugs}

We can classify most distributed system bugs in two categories: \emph{safety} and \emph{liveness} property violations~\cite{lamport1977proving}.

\begin{description}
\item[Safety] A safety property checks that an erroneous program state is \emph{never} reached, and is satisfied if it \emph{always} holds in each possible program execution.

\item[Liveness] A liveness property checks that some progress \emph{will} happen, and is satisfied if it \emph{always eventually} holds in each possible program execution.
\end{description}

\noindent
A safety property can be specified using an \emph{assertion} that fails if the property gets violated in some program state. An example of a generic safety property for message passing systems is to assert that whenever a message gets dequeued there must be an action that can handle the received message.

Liveness properties are much harder to specify and check since they apply over entire program executions and not just individual program states. Normally, liveness checking requires the identification of an infinite fair execution that never satisfies the liveness property~\cite{schuppan2004efficient, musuvathi2008fair}. Prior work~\cite{schuppan2004efficient} has proposed that assuming a program with finite state space, a liveness property can be converted into a safety property. Other researchers proposed the use of heuristics and only exploring finite executions of an infinite state space system using random walks to identify if a liveness property is violated~\cite{killian2007life}.


\section{Methodology}
\label{sec:method}

Our goal in this work is to \emph{test what is being executed}. Our approach involves using \psharp~\cite{deligiannis2015psharp}, a framework that provides: (i) an \emph{event-driven asynchronous programming} language for developing and modeling distributed systems; and (ii) a \emph{systematic concurrency testing} engine that can systematically explore all interleavings between asynchronous event handlers, as well as other nondeterministic events such as failures and timeouts.

\subsection{The \psharp framework}
\label{sec:method:psharp}

The \psharp language is an extension of \csharp, built on top of Microsoft's Roslyn\footnote{\url{https://github.com/dotnet/roslyn}} compiler, that enables asynchronous programming using communicating state-machines. \psharp machines can interact asynchronously by sending and receiving events,\footnote{We use the word ``event'' and ``message'' interchangeably.} an approach commonly used to develop distributed systems. This programming model is similar to actor-based approaches provided by other asynchronous programming languages (e.g. Scala~\cite{odersky2008programming} and Erlang~\cite{armstrong1996erlang}).

A \psharp machine consists of an input event queue, states, state transitions, event handlers, fields and methods. Machines run concurrently with each other, each executing an event handling loop that dequeues an event from the input queue and handles it by invoking an appropriate event handler. This handler might update a field, create a new machine, or send an event to another machine. In \psharp, a send operation is non-blocking; the message is simply enqueued into the input queue of the target machine, and it is up to the operating system scheduler to decide when to dequeue an event and handle it. All this functionality is provided in a lightweight runtime library, build on top of Microsoft's Task Parallel Library~\cite{leijen2009tpl}.

Because \psharp is built on top of \csharp, the programmer can blend \psharp and \csharp code; this not only lowers the overhead of learning a new language, but also allows \psharp to easily integrate with legacy code. Another advantage is that the programmer can use the familiar programming and debugging environment of Visual Studio.

A key capability of the \psharp runtime is that it can run in \emph{bug-finding mode}, where a embedded systematic testing engine captures and takes control of all sources of nondeterminism (such as event handler interleavings, failures, and client requests) in a \psharp program, and then systematically explores all possible executions to discover bugs.

\psharp is available as open-source\footnote{\url{https://github.com/p-org/PSharp}} and is currently used by various teams in Microsoft to develop and test distributed protocols and systems.

%The \psharp language belongs to the same family of languages as P~\cite{desai2013p}.

\subsection{Overview of our approach}
\label{sec:method:model}

In previous work~\cite{deligiannis2015psharp}, we approached the problem of testing legacy distributed systems as follows. First, we ported the system to \psharp, then we modeled its environment as \psharp state machines, and finally we tested the ported system and its environmental model using the \psharp systematic concurrency testing engine. The limitation of this approach is that it does not allow us to directly test a legacy system, as it has to be re-implemented first in \psharp. However, such endeavor is very costly and time consuming, and thus is not realistic for testing an existing production system, such as the Azure Storage vNext. Also, unless the code under test is the one that will actually execute, there is no guarantee that the real system will be bug-free.

To solve this problem, and allow \psharp to be used for testing legacy distributed systems, we decided to take a radically different approach. We provide the capability to model the environment of a system using \psharp, and then allow the developer to take advantage of existing language features, such as \emph{method dispatch}, to connect the system under test with the environmental model, and finally test it using the \psharp systematic concurrency testing engine.

We argue that our approach is \emph{flexible} since it allows the user to model \emph{as much} or \emph{as little} of the environment as required to achieve the desired level of testing. We also argue that our approach is \emph{generic} since a programmer can build on top of it to test more complicated use cases (see Section~\ref{}). Furthermore, the language features that are required to be used to connect the real code with the modeled code, are already being heavily used in production for testing purposes, which makes this method approachable to product groups.

\subsection{Modeling the environment}
\label{sec:method:model}

The environment of a distributed system might consist of other distributed systems and services, clients, operating system timers, as well as libraries for networking or other purposes. To be able to systematically test a distributed system, this environment must be modeled and all the interactions between the environment and the system, as well as all the nondeterminism, must be captured and controlled by the \psharp runtime.

\begin{figure}[t]
\centering
\includegraphics[width=\linewidth]{img/mocked_engine}
\caption{The real environment of the Extent Manager is replaced with a mocked version for testing.}
\label{fig:azurestoremodel}
\end{figure}

\subsubsection{Using method dispatch for modeling}
\label{sec:method:model:dd}

Method dispatch is the process of selecting which method, from a set of available methods with the same interface, should be invoked during a program's execution. There are two types of method dispatch: \emph{static}, which is resolved during compilation; and \emph{dynamic}, which is resolved in runtime.  \csharp (and thus \psharp) supports both static and dynamic dispatch, and provide the \texttt{virtual} modifier that can be used to declare a method which can be \emph{overridden} during runtime by an inheriting class. This capability is provided by the common language runtime (CLR) of Microsoft's .NET framework, and is a key feature of \csharp as well as other mainstream object-oriented languages.

Using method dispatch for modeling is straightforward. The system under test exposes a set of APIs as \emph{virtual methods}. The developer can then \emph{override} these APIs and replace them with \emph{mocks} that will execute instead of the original implementations during systematic testing with \psharp.

\begin{figure}[t]
\begin{lstlisting}
// Public interface of the real network engine
class NetworkEngine {
  public virtual void SendMessage(Socket s, Message msg);
  public virtual void EnqueueMessage(Message msg);
}

// The mocked network engine used during testing
class MockedNetEngine : NetworkEngine {
  ExtentManager EM; // Handle to actual system under test
  MachineId Env; // Handle to modeled environment
  
  public MockedNetEngine(ExtentManager em, MachineId env) {
    this.EM = em;
    this.Env = env;
  }
  
  public override void SendMessage(Socket s, Message msg) {
    PSharpRuntime.Send(this.Env, new MsgEvent(), s, msg);
  }
  
  public override void EnqueueMessage(Message msg) {
    this.EM.ProcessMessage(msg);
  }
}
\end{lstlisting}
\vspace{-2mm}
\caption{The mocked network engine used for testing the Azure Storage vNext system.}
\label{fig:enginecode}
%\vspace{-2mm}
\end{figure}

We now give an example of using dynamic dispatch to model the network engine of an extent manager in the Azure Storage vNext case study (see Figure~\ref{fig:enginecode}). The network engine is responsible for sending to and receiving messages from the various components of the system. During real execution, the network engine uses a custom remote procedure call (RPC) .NET library for communication. For testing, though, it is desirable to replace all calls to this RPC library with \psharp send and receive operations, which can be captured and systematically interleaved to find bugs. We easily achieved this by exposing the original send message operation of the network engine as a virtual method, and then overriding it for testing. In the overridden method, we created a \psharp event and then we wrapped the original message in this event's payload. Then, instead of invoking the RPC library, we invoke the \texttt{PSharpRuntime.Send(...)} method, which asynchronously sends the event (containing the original message) to the target extent node machine.

For mocking the receive operation, we take advantage of the implicit receive of events in \psharp machines. When a extent node machine receives an event, an appropriate event handler is invoked, which extracts the original message from the payload and then handles it accordingly.

\subsubsection{Abstracting timers}
\label{sec:method:model:timers}

Distributed systems are often using timers to determine when an event should be send from one component to another. For example, in the Azure Storage vNext system, each Extent Node is associated with a timer that fires of a synchronization message every 5 minutes and a heartbeat every 5 seconds. This timer is related to the liveness bug that we discovered: the synchronization message that gets fired every 5 minutes can potentially race with an Extent Node failure; if it arrives after the node failed, then the bug would manifest. Traditional testing techniques cannot easily find such a bug, due to the very infrequent occurrence of this race due to the timer. 

Our methodology in \psharp to systematically test distributed systems that rely on timers, is to abstract timers away, model them using message passing communication and introduce nondeterminism in their firing.

Figure~\ref{fig:timer} shows how we modeled a generic timer in the Azure Storage vNext case study. The Extend Manager, as well as each Extent Node in the harness, is associated with a unique \texttt{Timer} machine. When creating this machine, we pass as a payload the id of the machine that owns this timer. When the \texttt{Timer} machine is created, it stores this id in the \texttt{Owner} field and then transitions to the \texttt{Active} state. In this state, the \texttt{Timer} loops infinitely and nondeterministically sends a \texttt{TimerTickEvent} to \texttt{this.Owner}. When the Extent Node owner receives this event, it handles it by generating a synchronization message that is being send to the Extent Manager. Similarly, when the Extent Manager receives a \texttt{TimerTickEvent} from its own \texttt{Timer}, it handles it by nondeterministically invoking repair-related methods in the Extent Repair Center data structure. 

\begin{figure}[t]
\begin{lstlisting}
internal class Timer : Machine
{
  MachineId Owner; // Id of the owner machine

  [Start]
  [OnEntry(nameof(InitOnEntryAction))]
  [OnEventGotoState(typeof(Unit), typeof(Active))]
  class Init : MachineState { }

  void InitOnEntryAction()
  {
    this.Owner = (MachineId)this.Payload;
    // triggers state transition to Active
    this.Raise(new Unit());
  }

  [OnEntry(nameof(ProcessTickEvent))]
  [OnEventGotoState(typeof(Unit), typeof(Active))]
  class Active : MachineState { }

  void ProcessTickEvent()
  {
    // Nondeterministic boolean choice controlled by P#
    if (this.Nondet())
      // sends a timer tick event to the owner machine
      this.Send(this.Owner, new TimerTickEvent());
    // triggers state transition to Active
    this.Raise(new Unit());
  }
}
\end{lstlisting}
\vspace{-2mm}
\caption{Timers in Azure Storage vNext are modeled as a nondeterministic \psharp machines.}
\label{fig:timer}
%\vspace{-2mm}
\end{figure}

\subsubsection{Modeling and injecting failures}
\label{sec:method:model:failures}

In production, each Extent Node of the Azure Storage vNext system periodically (every 5 seconds) sends a heartbeat to the Extent Manager which notifies that the Extent Node is alive. Because we want to model failures and systematically inject them using \psharp, we abstract away the heartbeat mechanism in our harness. However, the Extent Manager logic relies on time intervals to detect node failures (see Figure~\ref{fig:expiration}). The way to abstract this time-related logic and connect the real code with the modeled code is to use virtual dispatch and override the virtual \texttt{IsNodeExpired} method with a mocked version.

\begin{figure}[t]
\begin{lstlisting}
// Real code for detecting node expiration
public virtual bool IsNodeExpired(string node_id, DateTime expiration)
{
  return DateTime.Compare(expiration, DateTime.Now) <= 0;
}

// Mocked code for detecting node expiration
public override bool IsNodeExpired(string node_id, DateTime expiration)
{
  return this.DeletedNodes.Contains(node_id);
}
\end{lstlisting}
\vspace{-2mm}
\caption{Abstracting the node expiration logic in the Extent Manager component of Azure Storage vNext.}
\label{fig:expiration}
%\vspace{-2mm}
\end{figure}

Figure~\ref{fig:expiration} presents how we mocked the node expiration detection method. Instead of comparing the time interval as in the original code, we now check if the set \texttt{DeletedNodes} contains the id of the Extent Node that we are checking for expiration. If it contains the id, then it means that the node has failed. The \texttt{Environment} machine that we have created as part of our testing \psharp harness, will nondeterministically choose a node to kill, then send the id of this killed node to the Extent Manager wrapper machine, who will in turn add it to the \texttt{DeletedNodes} set.

\subsection{Handling intra-machine concurrency}
\label{sec:method:async}

Async/Await

custom schedulers, etc

\subsection{Other}
\label{sec:method:other}

Logs/traces -> user can extend them?

\PDComment{mention dependency injection pattern?}


\section{Case studies}
\label{sec:cases}

% cheng to review
\subsection{Distributed Extent Management System}
\label{sec:cases:azurestore}

We used \psharp to test the \emph{distributed extent management} component of the Windows Azure vNext distributed storage system. This component is responsible for managing the partitioned extent metadata and works as follows.

There are multiple \emph{extent managers} (EMgrs) that are in charge of managing a subset of the extents. Each EMgr communicates asynchronously (via remote procedure calls) with a number of \emph{extent nodes} (ENs) that store the extents. Each EN sends: (i) a \emph{heartbeat} every 5 seconds, to notify the EMgr that it is still available; and (ii) a \emph{sync report} every 5 minutes, to synchronize its extent with the rest of the nodes. The EMgr contains two data structures: an \emph{extent center} (ECtr), which is updated every time an EN syncs; and an EN map, which is updated every time it receives a heartbeat from an EN. The EN map runs an EN expiration loop that is responsible for removing ENs that have expired from the EN map and also delete them from the ECtr. Finally, the EMgr runs an extent repair loop, which examines all contents of all extents in the ECtr and schedules repairs of extents if they are out-of-date (via a repair request).

\subsection{Live Azure Table Migration}

We also used \psharp to test the MigratingTable library, which is capable of transparently migrating a data set between tables in the Windows Azure storage service while an application is accessing the data set.  MigratingTable provides a ``virtual table'' with an API similar to that of an ordinary Azure table, backed by a pair of ``old'' and ``new''  tables.  It moves all data from the old table to the new table in the background.  Meanwhile, each read or write issued to the virtual table is translated to a sequence of reads and writes on the backend tables according to a protocol we designed, which guarantees linearizability of operations on the virtual table across multiple application processes assuming that the backend tables respect their own linearizability guarantees.

% N.B. Artifact Services is mentioned at http://research.microsoft.com/en-us/people/schulte/.  Hopefully it's OK to reveal that it was the system in this case study. ~ Matt 2015-08-17
The initial motivation for MigratingTable was to solve a scaling problem for Artifact Services, an internal Microsoft system with a data set that is sharded across tables in different Azure storage accounts because of the limit on traffic supported by a single Azure storage account.  As the traffic continues to grow over time, we will need to reshard the data set across a greater number of Azure storage accounts without interrupting service.  During such a resharding, our sharding manager will identify each key range that should migrate to a different table, and we will use a separate MigratingTable instance for each such key range to actually perform the migration.  MigratingTable may also be useful to migrate data to a table with different values of configuration parameters that Azure does not support changing on an existing table, such as geographic location.

Since we were designing a new concurrent protocol that we expected to become increasingly complex over time as we add optimizations, we planned from the beginning to maintain a \psharp test harness along with the protocol to maintain confidence in its correctness.

\subsubsection{Modeling approach}
% TBD where this ends up in the paper ~ Matt 2015-08-17

MigratingTable implements an interface called IChainTable2, which provides the core read and write functionality of the real Azure table API with one exception: it provides a weaker consistency property for multi-page reads, since the original property would have been difficult to achieve for no benefit to applications we could foresee.  MigratingTable requires that its backend tables also implement IChainTable2, and we wrote a simple adapter to expose physical Azure tables as IChainTable2.  Our goal was then to verify that when multiple application processes issue ``input'' read and write calls to their own MigratingTable instances backed by the same tables, the behavior is compliant with the specification of IChainTable2 for the combined input history.

% N.B. SpecTable = InMemoryTableWithHistory in the current codebase. ~ Matt 2015-08-17
Since the specification is deterministic under sequential calls except for the results of multi-page reads, we decided the easiest way to formulate it for automated testing was to write an in-memory reference implementation called SpecTable.  Given a multi-page read, SpecTable can actually produce a list of all valid results.  Our correctness property is then:
\begin{quote}
For every execution trace of a collection of MigratingTables backed by the same pair of SpecTables (which nondeterministically choose one of the valid results for each multi-page read), there exists a linearization of the combined input history such that the output in the original trace matches the output of a ``reference'' SpecTable on the linearized input.
\end{quote}
%
\def\term#1{\emph{#1}}
We instrumented MigratingTable to report the \term{linearization point} of each input call, which in our case is always one of the corresponding \term{backend calls} to the backend tables (often the last).  Specifically, after each backend call, MigratingTable reports whether it was the linearization point, which may depend on the result of the call.  This makes it possible to verify the correctness property as the system executes.  We have a \term{tables machine} containing all three SpecTables and a collection of \term{service machines} each containing a MigratingTable.  Each service machine issues a random sequence of input calls to its MigratingTable, which sends backend calls to the tables machine.  When MigratingTable reports the linearization point of an input call, the service machine sends that input call to the reference table.  When an input call completes, the service machine checks that the results from the MigratingTable and the reference table agree.  \psharp controls the interleaving of the backend calls.  After the tables machine processes a backend call, it enters a state that defers further backend calls until MigratingTable has reported whether the backend call was a linearization point and (if so) the call to the reference table has been made.  We use the \psharp nondeterminism API to generate the input calls, so in principle an exhaustive \psharp behavior exploration strategy such as DFS could be used to exhaustively test MigratingTable up to some bound.

% Draft
We wanted to implement the core MigratingTable algorithms in \csharp ``async'' code, like most of Artifact Services.  Async code is readable like traditional procedural code but is translated by the compiler to an event-driven state machine using the Task Parallel Library, gaining most of the performance benefits of that style.  By default, all event processing occurs on the .NET thread pool.

We had to arrange for the continuations to execute within the context of a single \psharp machine, which we did by installing a custom SynchronizationContext.  Then we implemented async RPC between machines in terms of message passing, using RealProxy to avoid most of the marshaling boilerplate.

\section{Evaluation}
\label{sec:eval}

Paragraph that says what will be discussed in evaluation and what are the results (very short)

\subsection{Experimental Setup}

\subsection{Benchmarks}

Benchmarks that are not production code (e.g. multipaxos) that can be used to evaluate testing

Cheng's case study with the actual bug

Matt's case study with injected bugs

\subsection{Stressing testing vs Systematic testing}

Comparison of traditional stress testing VS what we do with \psharp

How long you had to run the test to find the bug (if you can find it) and how long are the traces

How easy it is to pinpoint the error using the \psharp traces, comparing to traditional stress testing

Effort required to use the system

\section{Related Work}
\label{sec:rw}

Most related to our work are model checking~\cite{godefroid1997verisoft} and systematic concurrency testing~\cite{musuvathi2008finding, emmi2011delay, thomson2014sct}, two powerful techniques that have been widely used in the past for finding Heisenbugs in the actual implementation of distributed systems~\cite{killian2007life, yang2009modist, yabandeh2009crystalball, guerraoui2011model, guo2011practical, simsa2011dbug, gunawi2011fate, leesatapornwongsa2014samc}.

State-of-the-art model checkers, such as \textsc{MoDist}~\cite{yang2009modist} and dBug~\cite{simsa2011dbug}, typically focus on testing entire, often \emph{unmodified}, distributed systems, an approach that easily leads to state-space explosion. \textsc{DeMeter}~\cite{guo2011practical}, built on top of \textsc{MoDist}, aims to reduce the state-space when exploring unmodified distributed systems. \textsc{DeMeter} explores individual components of a large system in isolation, and then dynamically extracts interface behavior between components to perform a global exploration. In contrast, we try to offer a more pragmatic approach for handling state-space explosion. We first \emph{partially} model a distributed system using \psharp. Then, we systematically test the actual implementation of each system component against its \psharp test harness. Our approach aims to enhance unit and integration testing, techniques widely used in production, where only individual or a small number of components are tested at each time.

SAMC~\cite{leesatapornwongsa2014samc} offers a way of incorporating application-specific information during systematic testing to reduce the set of interleavings that the tool has to explore. Such techniques based on partial-order reduction~\cite{godefroid1996partial, flanagan2005dynamic} are complementary to our approach: \psharp could use them to reduce the exploration state-space. Likewise, other tools can use language technology like \psharp to write models and reduce the complexity of the system-under-test.

\textsc{MaceMc}~\cite{killian2007life} is a model checker for distributed systems written in the \textsc{Mace}~\cite{killian2007mace} language. The focus of \textsc{MaceMc} is to find liveness property violations using an algorithm based on depth-bounded random walk and the use of heuristics. Because \textsc{MaceMc} can only test systems written in \textsc{Mace}, it cannot be easily used in an industrial setting. In contrast, \psharp can be applied on legacy code written in \csharp, a mainstream language.

%\textsc{Fate} and \textsc{Destini} is a framework for systematically injecting failures in distributed systems~\cite{gunawi2011fate}. This framework focuses in exercising various failure scenarios, whereas \psharp can be used for testing generic safety and liveness properties of distributed systems.

Formal methods have been successfully used in industry to verify the correctness of distributed protocols. A recent notable example is the use of TLA+~\cite{lamport1994temporal} by the Amazon Web Services team~\cite{newcombe2015aws}. TLA+ is an expressive formal specification language that can be used to design and verify concurrent programs via model checking. A limitation of TLA+, as well as other similar specification languages, is that they are applied on a model of the system and not the actual system. Even if the model is verified, the gap between the real-world implementation and the verified model is still significant, so implementation bugs are still a realistic concern.

Another formal approach is Verdi~\cite{wilcox2015verdi}, a framework for writing and verifying distributed systems in Coq~\cite{barras1997coq}. Verdi generates OCaml code from the verifying system, which can be used for execution. In contrast, \psharp performs bounded testing on a system already written in \csharp, which in our experience lowers the bar for adoption by engineering teams. Verdi does not currently support detecting liveness property violations, an important class of bugs in distributed storage systems.


\section{Conclusion}
\label{sec:concl}

Draft.

%\section{Acknowledgments}
%Draft.

%
% The following two commands are all you need in the
% initial runs of your .tex file to
% produce the bibliography for the citations in your paper.
\bibliographystyle{abbrv}
\bibliography{references}  % sigproc.bib is the name of the Bibliography in this case
% You must have a proper ".bib" file
%  and remember to run:
% latex bibtex latex latex
% to resolve all references
%
% ACM needs 'a single self-contained file'!
%
%APPENDICES are optional
%\balancecolumns

\end{document}
