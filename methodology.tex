Our goal in this work is to \emph{test what is being executed}. Our approach involves using \psharp~\cite{deligiannis2015psharp}, a framework that provides: (i) an \emph{event-driven asynchronous programming} language for developing and modeling distributed systems; and (ii) a \emph{systematic concurrency testing} engine that can systematically explore all interleavings between asynchronous event handlers, as well as other nondeterministic events such as failures and timeouts.

\subsection{The \psharp framework}
\label{sec:method:psharp}

The \psharp language is an extension of \csharp, built on top of Microsoft's Roslyn\footnote{\url{https://github.com/dotnet/roslyn}} compiler, that enables asynchronous programming using communicating state-machines. \psharp machines can interact asynchronously by sending and receiving events,\footnote{We use the word ``event'' and ``message'' interchangeably.} an approach commonly used to develop distributed systems. This programming model is similar to actor-based approaches provided by other asynchronous programming languages (e.g. Scala~\cite{odersky2008programming} and Erlang~\cite{armstrong1996erlang}).

A \psharp machine consists of an input event queue, states, state transitions, event handlers, fields and methods. Machines run concurrently with each other, each executing an event handling loop that dequeues an event from the input queue and handles it by invoking an appropriate event handler. This handler might update a field, create a new machine, or send an event to another machine. In \psharp, a send operation is non-blocking; the message is simply enqueued into the input queue of the target machine, and it is up to the operating system scheduler to decide when to dequeue an event and handle it. All this functionality is provided in a lightweight runtime library, build on top of Microsoft's Task Parallel Library~\cite{leijen2009tpl}.

Because \psharp is built on top of \csharp, the programmer can blend \psharp and \csharp code; this not only lowers the overhead of learning a new language, but also allows \psharp to easily integrate with legacy code. Another advantage is that the programmer can use the familiar programming and debugging environment of Visual Studio.

A key capability of the \psharp runtime is that it can run in \emph{bug-finding mode}, where a embedded systematic testing engine captures and takes control of all sources of nondeterminism (such as event handler interleavings, failures, and client requests) in a \psharp program, and then systematically explores all possible executions to discover bugs.

\psharp is available as open-source\footnote{\url{https://github.com/p-org/PSharp}} and is currently used by various teams in Microsoft to develop and test distributed protocols and systems.

%The \psharp language belongs to the same family of languages as P~\cite{desai2013p}.

\subsection{Overview of our approach}
\label{sec:method:model}

In previous work~\cite{deligiannis2015psharp}, we approached the problem of testing legacy distributed systems as follows. First, we ported the system to \psharp, then we modeled its environment as \psharp state machines, and finally we tested the ported system and its environmental model using the \psharp systematic concurrency testing engine. The limitation of this approach is that it does not allow us to directly test a legacy system, as it has to be re-implemented first in \psharp. However, such endeavor is very costly and time consuming, and thus is not realistic for testing an existing production system, such as the Azure Storage vNext. Also, unless the code under test is the one that will actually execute, there is no guarantee that the real system will be bug-free.

To solve this problem, and allow \psharp to be used for testing legacy distributed systems, we decided to take a radically different approach. We provide the capability to model the environment of a system using \psharp, and then allow the developer to take advantage of existing language features, such as \emph{method dispatch}, to connect the system under test with the environmental model, and finally test it using the \psharp systematic concurrency testing engine.

We argue that our approach is \emph{flexible} since it allows the user to model \emph{as much} or \emph{as little} of the environment as required to achieve the desired level of testing. We also argue that our approach is \emph{generic} since a programmer can build on top of it to test more complicated use cases (see Section~\ref{}). Furthermore, the language features that are required to be used to connect the real code with the modeled code, are already being heavily used in production for testing purposes, which makes this method approachable to product groups.

\subsection{Modeling the environment}
\label{sec:method:model}

The environment of a distributed system might consist of other distributed systems and services, clients, operating system timers, as well as libraries for networking or other purposes. To be able to systematically test a distributed system, this environment must be modeled and all the interactions between the environment and the system, as well as all the nondeterminism, must be captured and controlled by the \psharp runtime.

\begin{figure}[t]
\centering
\includegraphics[width=\linewidth]{img/mocked_engine}
\caption{The real environment of the Extent Manager is replaced with a mocked version for testing.}
\label{fig:azurestoremodel}
\end{figure}

\subsubsection{Using method dispatch for modeling}
\label{sec:method:model:dd}

Method dispatch is the process of selecting which method, from a set of available methods with the same interface, should be invoked during a program's execution. There are two types of method dispatch: \emph{static}, which is resolved during compilation; and \emph{dynamic}, which is resolved in runtime.  \csharp (and thus \psharp) supports both static and dynamic dispatch, and provide the \texttt{virtual} modifier that can be used to declare a method which can be \emph{overridden} during runtime by an inheriting class. This capability is provided by the common language runtime (CLR) of Microsoft's .NET framework, and is a key feature of \csharp as well as other mainstream object-oriented languages.

Using method dispatch for modeling is straightforward. The system under test exposes a set of APIs as \emph{virtual methods}. The developer can then \emph{override} these APIs and replace them with \emph{mocks} that will execute instead of the original implementations during systematic testing with \psharp.

\begin{figure}[t]
\begin{lstlisting}
// Public interface of the real network engine
class NetworkEngine {
  public virtual void SendMessage(Socket s, Message msg);
  public virtual void EnqueueMessage(Message msg);
}

// The mocked network engine used during testing
class MockedNetEngine : NetworkEngine {
  ExtentManager EM; // Handle to actual system under test
  MachineId Env; // Handle to modeled environment
  
  public MockedNetEngine(ExtentManager em, MachineId env) {
    this.EM = em;
    this.Env = env;
  }
  
  public override void SendMessage(Socket s, Message msg) {
    PSharpRuntime.Send(this.Env, new MsgEvent(), s, msg);
  }
  
  public override void EnqueueMessage(Message msg) {
    this.EM.ProcessMessage(msg);
  }
}
\end{lstlisting}
\vspace{-2mm}
\caption{The mocked network engine used for testing the Azure Storage vNext system.}
\label{fig:enginecode}
%\vspace{-2mm}
\end{figure}

We now give an example of using dynamic dispatch to model the network engine of an extent manager in the Azure Storage vNext case study (see Figure~\ref{fig:enginecode}). The network engine is responsible for sending to and receiving messages from the various components of the system. During real execution, the network engine uses a custom remote procedure call (RPC) .NET library for communication. For testing, though, it is desirable to replace all calls to this RPC library with \psharp send and receive operations, which can be captured and systematically interleaved to find bugs. We easily achieved this by exposing the original send message operation of the network engine as a virtual method, and then overriding it for testing. In the overridden method, we created a \psharp event and then we wrapped the original message in this event's payload. Then, instead of invoking the RPC library, we invoke the \texttt{PSharpRuntime.Send(...)} method, which asynchronously sends the event (containing the original message) to the target extent node machine.

For mocking the receive operation, we take advantage of the implicit receive of events in \psharp machines. When a extent node machine receives an event, an appropriate event handler is invoked, which extracts the original message from the payload and then handles it accordingly.

\subsubsection{Abstracting timers}
\label{sec:method:model:timers}

Distributed systems are often using timers to determine when an event should be send from one component to another. For example, in the Azure Storage vNext system, each Extent Node is associated with a timer that fires of a synchronization message every 5 minutes and a heartbeat every 5 seconds. This timer is related to the liveness bug that we discovered: the synchronization message that gets fired every 5 minutes can potentially race with an Extent Node failure; if it arrives after the node failed, then the bug would manifest. Traditional testing techniques cannot easily find such a bug, due to the very infrequent occurrence of this race due to the timer. 

Our methodology in \psharp to systematically test distributed systems that rely on timers, is to abstract timers away, model them using message passing communication and introduce nondeterminism in their firing.

Figure~\ref{fig:timer} shows how we modeled a generic timer in the Azure Storage vNext case study. The Extend Manager, as well as each Extent Node in the harness, is associated with a unique \texttt{Timer} machine. When creating this machine, we pass as a payload the id of the machine that owns this timer. When the \texttt{Timer} machine is created, it stores this id in the \texttt{Owner} field and then transitions to the \texttt{Active} state. In this state, the \texttt{Timer} loops infinitely and nondeterministically sends a \texttt{TimerTickEvent} to \texttt{this.Owner}. When the Extent Node owner receives this event, it handles it by generating a synchronization message that is being send to the Extent Manager. Similarly, when the Extent Manager receives a \texttt{TimerTickEvent} from its own \texttt{Timer}, it handles it by nondeterministically invoking repair-related methods in the Extent Repair Center data structure. 

\begin{figure}[t]
\begin{lstlisting}
internal class Timer : Machine
{
  MachineId Owner; // Id of the owner machine

  [Start]
  [OnEntry(nameof(InitOnEntryAction))]
  [OnEventGotoState(typeof(Unit), typeof(Active))]
  class Init : MachineState { }

  void InitOnEntryAction()
  {
    this.Owner = (MachineId)this.Payload;
    // triggers state transition to Active
    this.Raise(new Unit());
  }

  [OnEntry(nameof(ProcessTickEvent))]
  [OnEventGotoState(typeof(Unit), typeof(Active))]
  class Active : MachineState { }

  void ProcessTickEvent()
  {
    // Nondeterministic boolean choice controlled by P#
    if (this.Nondet())
      // sends a timer tick event to the owner machine
      this.Send(this.Owner, new TimerTickEvent());
    // triggers state transition to Active
    this.Raise(new Unit());
  }
}
\end{lstlisting}
\vspace{-2mm}
\caption{Timers in Azure Storage vNext are modeled as a nondeterministic \psharp machines.}
\label{fig:timer}
%\vspace{-2mm}
\end{figure}

\subsubsection{Modeling and injecting failures}
\label{sec:method:model:failures}

In production, each Extent Node of the Azure Storage vNext system periodically (every 5 seconds) sends a heartbeat to the Extent Manager which notifies that the Extent Node is alive. Because we want to model failures and systematically inject them using \psharp, we abstract away the heartbeat mechanism in our harness. However, the Extent Manager logic relies on time intervals to detect node failures (see Figure~\ref{fig:expiration}). The way to abstract this time-related logic and connect the real code with the modeled code is to use virtual dispatch and override the virtual \texttt{IsNodeExpired} method with a mocked version.

\begin{figure}[t]
\begin{lstlisting}
// Real code for detecting node expiration
public virtual bool IsNodeExpired(string node_id, DateTime expiration)
{
  return DateTime.Compare(expiration, DateTime.Now) <= 0;
}

// Mocked code for detecting node expiration
public override bool IsNodeExpired(string node_id, DateTime expiration)
{
  return this.DeletedNodes.Contains(node_id);
}
\end{lstlisting}
\vspace{-2mm}
\caption{Abstracting the node expiration logic in the Extent Manager component of Azure Storage vNext.}
\label{fig:expiration}
%\vspace{-2mm}
\end{figure}

Figure~\ref{fig:expiration} presents how we mocked the node expiration detection method. Instead of comparing the time interval as in the original code, we now check if the set \texttt{DeletedNodes} contains the id of the Extent Node that we are checking for expiration. If it contains the id, then it means that the node has failed. The \texttt{Environment} machine that we have created as part of our testing \psharp harness, will nondeterministically choose a node to kill, then send the id of this killed node to the Extent Manager wrapper machine, who will in turn add it to the \texttt{DeletedNodes} set.

\subsection{Handling intra-machine concurrency}
\label{sec:method:async}

Async/Await

custom schedulers, etc

\subsection{Other}
\label{sec:method:other}

Logs/traces -> user can extend them?

\PDComment{mention dependency injection pattern?}
