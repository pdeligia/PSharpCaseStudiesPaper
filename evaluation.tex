We compared the bug-finding effectiveness of \psharp systematic concurrency testing to traditional ``stress testing'' without systematic concurrency control on several benchmark bugs, described below.  We report the length of time it took for the test to fail and remark on the effort needed to diagnose the bug from the test output. \MMComment{Overview of actual results?}

\subsection{Experimental Setup}

All experiments were run on \MMComment{insert MSR machine, or perhaps CloudDevR1S04 if John/Matt are able to grant access to the rest of the collaborators}. \MMComment{specs}

\subsection{Benchmarks}

\newcommand{\colspacing}{\hspace{1.8em}}
\begin{table}[t]
\small
\centering
\setlength{\tabcolsep}{0.3em}
\label{tab:stats}
\begin{tabular}{l rrrrr rr}
\centering
& \multicolumn{2}{c}{\textbf{System}}
& \multicolumn{4}{c}{\textbf{\psharp Model}}\\
\cmidrule(lr){2-3}
\cmidrule(lr){4-7}

\textbf{System under Test}
& \multicolumn{1}{r}{\textbf{\#LoC}}
& \multicolumn{1}{r}{\textbf{\#B}}
& \multicolumn{1}{r}{\textbf{\#LoC}}
& \multicolumn{1}{r}{\textbf{\#M}}
& \multicolumn{1}{r}{\textbf{\#ST}}
& \multicolumn{1}{r}{\textbf{\#AH}}\\[0.3em]

\toprule

vNext Extent Manager
& \multicolumn{1}{r}{0}
& \multicolumn{1}{r}{1}
& \multicolumn{1}{r}{684}
& \multicolumn{1}{r}{5}
& \multicolumn{1}{r}{11}
& \multicolumn{1}{r}{17}\\

MigratingTable
% These numbers should be taken with a grain of salt because I probably have a
% higher mean and variance in amount of comments added about code design issues
% than an average software engineer. ~ Matt
& \multicolumn{1}{r}{2267}
& \multicolumn{1}{r}{11}
& \multicolumn{1}{r}{2275}
% These numbers too.  There's plenty of complexity in MigratingTable, but it's
% in the payloads of a small number of event types. ~ Matt
& \multicolumn{1}{r}{3}
& \multicolumn{1}{r}{5}
& \multicolumn{1}{r}{10}\\

Fabric User Service
& \multicolumn{1}{r}{31959}
& \multicolumn{1}{r}{0}
& \multicolumn{1}{r}{6534}
& \multicolumn{1}{r}{13}
& \multicolumn{1}{r}{21}
& \multicolumn{1}{r}{87}\\

\bottomrule

\end{tabular}
\caption{Statistics.}
%\vspace{-3mm}
\end{table}

\PDComment{We probably do not need this now that we have many bugs from the case studies -- benchmarks that are not production code (e.g. multipaxos) that can be used to evaluate testing}

Cheng's case study with the actual bug

After fixing all the bugs we found in MigratingTable, we added an option to conditionally reintroduce each of the following bugs:
\begin{enumerate}
\item Filtering on a user-defined property did not retrieve non-matching rows from the new table that might shadow a matching row from the old table.
\item ...
\end{enumerate}

\subsection{Stress testing vs \psharp testing}

\setlength{\tabcolsep}{.42em}
\begin{table*}[t]
\small
\centering
\begin{tabular}{l rr rrrr}
\centering
& & \multicolumn{4}{c}{\textbf{\psharp Systematic Testing (Random)}}
& \multicolumn{4}{c}{\textbf{\psharp Systematic Testing (PCT)}}\\
\cmidrule(lr){3-6}
\cmidrule(lr){7-10}

& & \multirow{1}{*}{\textbf{Time to}}
& &
& & \multirow{1}{*}{\textbf{Time to}}
& & &\\

\textbf{CS}
& \textbf{Bug Identifier}
& \multicolumn{1}{r}{\textbf{Bug (s)}}
& \multicolumn{1}{r}{\textbf{\#SP}}
& \multicolumn{1}{r}{\textbf{\%Buggy}}
& \multicolumn{1}{r}{\textbf{BF?}}
& \multicolumn{1}{r}{\textbf{Bug (s)}}
& \multicolumn{1}{r}{\textbf{\#SP}}
& \multicolumn{1}{r}{\textbf{\%Buggy}}
& \multicolumn{1}{r}{\textbf{BF?}}\\[0.3em]

\toprule

1
& ExtentNodeLivenessViolation

& \multicolumn{1}{r}{}
& \multicolumn{1}{r}{}
& \multicolumn{1}{r}{x\%}
& \multicolumn{1}{r}{\cmark}

& \multicolumn{1}{r}{}
& \multicolumn{1}{r}{}
& \multicolumn{1}{r}{x\%}
& \multicolumn{1}{r}{\cmark}\\

\midrule

2
& QueryAtomicFilterShadowing

& \multicolumn{1}{r}{157.22}
& \multicolumn{1}{r}{165}
& \multicolumn{1}{r}{0.018\%}
& \multicolumn{1}{r}{\cmark}

& \multicolumn{1}{r}{350.46}
& \multicolumn{1}{r}{108}
& \multicolumn{1}{r}{0.220\%}
& \multicolumn{1}{r}{\cmark}\\

2
& QueryStreamedLock

& \multicolumn{1}{r}{2121.45}
& \multicolumn{1}{r}{181}
& \multicolumn{1}{r}{0.001\%}
& \multicolumn{1}{r}{\cmark}

& \multicolumn{1}{r}{6.58}
& \multicolumn{1}{r}{220}
& \multicolumn{1}{r}{0.664\%}
& \multicolumn{1}{r}{\cmark}\\

2
& QueryStreamedBackUpNewStream

& \multicolumn{1}{r}{-}
& \multicolumn{1}{r}{-}
& \multicolumn{1}{r}{-}
& \multicolumn{1}{r}{\xmark}

& \multicolumn{1}{r}{5.95}
& \multicolumn{1}{r}{232}
& \multicolumn{1}{r}{0.458\%}
& \multicolumn{1}{r}{\cmark}\\

2
& DeleteNoLeaveTombstonesEtag

& \multicolumn{1}{r}{-}
& \multicolumn{1}{r}{-}
& \multicolumn{1}{r}{-}
& \multicolumn{1}{r}{\xmark}

& \multicolumn{1}{r}{4.69}
& \multicolumn{1}{r}{272}
& \multicolumn{1}{r}{1.187\%}
& \multicolumn{1}{r}{\cmark}\\

2
& DeletePrimaryKey

& \multicolumn{1}{r}{2.72}
& \multicolumn{1}{r}{168}
& \multicolumn{1}{r}{3.895\%}
& \multicolumn{1}{r}{\cmark}

& \multicolumn{1}{r}{2.37}
& \multicolumn{1}{r}{171}
& \multicolumn{1}{r}{4.137\%}
& \multicolumn{1}{r}{\cmark}\\

2
& EnsurePartitionSwitchedFromPopulated

& \multicolumn{1}{r}{25.17}
& \multicolumn{1}{r}{85}
& \multicolumn{1}{r}{0.076\%}
& \multicolumn{1}{r}{\cmark}

& \multicolumn{1}{r}{1.57}
& \multicolumn{1}{r}{136}
& \multicolumn{1}{r}{1.491\%}
& \multicolumn{1}{r}{\cmark}\\

2
& TombstoneOutputETag

& \multicolumn{1}{r}{8.25}
& \multicolumn{1}{r}{305}
& \multicolumn{1}{r}{0.370\%}
& \multicolumn{1}{r}{\cmark}

& \multicolumn{1}{r}{3.40}
& \multicolumn{1}{r}{242}
& \multicolumn{1}{r}{0.473\%}
& \multicolumn{1}{r}{\cmark}\\

2
& DeleteIfExistsNotLinearizable

& \multicolumn{1}{r}{-}
& \multicolumn{1}{r}{-}
& \multicolumn{1}{r}{-}
& \multicolumn{1}{r}{\xmark}

& \multicolumn{1}{r}{3.19}
& \multicolumn{1}{r}{242}
& \multicolumn{1}{r}{0.195\%}
& \multicolumn{1}{r}{\cmark}\\

\midrule

$\diamond$2
& QueryStreamedFilterShadowing

& \multicolumn{1}{r}{0.55}
& \multicolumn{1}{r}{79}
& \multicolumn{1}{r}{24.908\%}
& \multicolumn{1}{r}{\cmark}

& \multicolumn{1}{r}{0.41}
& \multicolumn{1}{r}{79}
& \multicolumn{1}{r}{24.998\%}
& \multicolumn{1}{r}{\cmark}\\

$\diamond$2
& MigrateSkipPreferOld

& \multicolumn{1}{r}{-}
& \multicolumn{1}{r}{-}
& \multicolumn{1}{r}{-}
& \multicolumn{1}{r}{\xmark}

& \multicolumn{1}{r}{1.13}
& \multicolumn{1}{r}{115}
& \multicolumn{1}{r}{25.043\%}
& \multicolumn{1}{r}{\cmark}\\

$\diamond$2
& MigrateSkipUseNewWithTombstones

& \multicolumn{1}{r}{-}
& \multicolumn{1}{r}{-}
& \multicolumn{1}{r}{-}
& \multicolumn{1}{r}{\xmark}

& \multicolumn{1}{r}{1.16}
& \multicolumn{1}{r}{120}
& \multicolumn{1}{r}{x\%}
& \multicolumn{1}{r}{\cmark}\\

$\diamond$2
& InsertBehindMigrator

& \multicolumn{1}{r}{0.32}
& \multicolumn{1}{r}{47}
& \multicolumn{1}{r}{72.534\%}
& \multicolumn{1}{r}{\cmark}

& \multicolumn{1}{r}{0.31}
& \multicolumn{1}{r}{47}
& \multicolumn{1}{r}{49.985\%}
& \multicolumn{1}{r}{\cmark}\\[0.1em]

\end{tabular}
\caption{Results from 100000 iterations of \psharp random walk scheduler.}
\label{tab:testing}
\end{table*}

Comparison of traditional stress testing VS what we do with \psharp

How long you had to run the test to find the bug (if you can find it) and how long are the traces

How easy it is to pinpoint the error using the \psharp traces, comparing to traditional stress testing

Effort required to use the system

Controlled random scheduling has proven to be efficient for finding concurrency bugs~\cite{thomson2014sct, deligiannis2015psharp}.
