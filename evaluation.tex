We compared the bug-finding effectiveness of \psharp systematic concurrency testing to traditional ``stress testing'' without systematic concurrency control on several benchmark bugs, described below.  We report the length of time it took for the test to fail and remark on the effort needed to diagnose the bug from the test output. \MMComment{Overview of actual results?}

\subsection{Experimental Setup}

We performed all experiments using the Windows PowerShell tool on a 2.50GHz Intel Core i5-4300U CPU with 8GB RAM running Windows 10 Pro 64-bit.

Because we fixed all bugs that we discovered using the \psharp systematic testing engine in the case studies, for the evaluation we added an option to conditionally reintroduce each one of the bugs so that we can measure how fast and how often each bug can be discovered with the \psharp random and the PCT scheduler.

Cheng's case study with the actual bug

After fixing all the bugs we found in MigratingTable, we added an option to conditionally reintroduce each of the following bugs:
\begin{enumerate}
\item Filtering on a user-defined property did not retrieve non-matching rows from the new table that might shadow a matching row from the old table.
\item ...
\end{enumerate}

\newcommand{\colspacing}{\hspace{1.8em}}
\begin{table}[t]
\small
\centering
\setlength{\tabcolsep}{0.3em}
\label{tab:stats}
\begin{tabular}{l rrrrr rr}
\centering
& \multicolumn{2}{c}{\textbf{\#LoC}}
& \multicolumn{3}{c}{\textbf{\psharp statistics}}
& \multirow{2}{*}{\textbf{\#Bugs}}\\
\cmidrule(lr){2-3}
\cmidrule(lr){4-6}

\textbf{Azure Systems}
& \multicolumn{1}{r}{\textbf{Real}}
& \multicolumn{1}{r}{\textbf{Model}}
& \textbf{\#M}
& \textbf{\#ST}
& \textbf{\#AB}
& \textbf{found}\\[0.3em]

\toprule

Azure Storage vNext
& \multicolumn{1}{r}{0}
& \multicolumn{1}{r}{684}
& \multicolumn{1}{r}{5}
& \multicolumn{1}{r}{11}
& \multicolumn{1}{r}{17}
& \multicolumn{1}{r}{1}\\

Live Table Migration
& \multicolumn{1}{r}{0}
& \multicolumn{1}{r}{0}
& \multicolumn{1}{r}{0}
& \multicolumn{1}{r}{0}
& \multicolumn{1}{r}{0}
& \multicolumn{1}{r}{$>$10}\\

Fabric User Service
& \multicolumn{1}{r}{0}
& \multicolumn{1}{r}{0}
& \multicolumn{1}{r}{0}
& \multicolumn{1}{r}{0}
& \multicolumn{1}{r}{0}
& \multicolumn{1}{r}{0}\\

\bottomrule
\end{tabular}
\caption{Statistics.}
%\vspace{-3mm}
\end{table}

\subsection{Systematic testing with \psharp}

\setlength{\tabcolsep}{.72em}
\begin{table*}[t]
\small
\centering
\begin{tabular}{rl rrr rrr}
\centering
& \multicolumn{2}{c}{\textbf{Stress Testing}}
& \multicolumn{5}{c}{\textbf{\psharp Testing (Random Scheduler)}}\\
\cmidrule(lr){2-3}
\cmidrule(lr){4-8}


&
& \multirow{1}{*}{\textbf{Bug}}
& \multirow{1}{*}{\textbf{Time to}}
& \multirow{1}{*}{\textbf{Total}}
& &
& \multirow{1}{*}{\textbf{Bug}} \\

\textbf{Bugs}
& \multicolumn{1}{r}{\textbf{Time (s)}}
& \multicolumn{1}{r}{\textbf{found?}}

& \multicolumn{1}{r}{\textbf{Bug (s)}}
& \multicolumn{1}{r}{\textbf{Time (s)}}
& \multicolumn{1}{r}{\textbf{\#SP}}
& \multicolumn{1}{r}{\textbf{\%Buggy}}
& \multicolumn{1}{r}{\textbf{found?}}\\[0.3em]

\toprule

vNext Liveness Bug

& \multicolumn{1}{r}{0}
& \multicolumn{1}{r}{\xmark}

& \multicolumn{1}{r}{}
& \multicolumn{1}{r}{}
& \multicolumn{1}{r}{}
& \multicolumn{1}{r}{6\%}
& \multicolumn{1}{r}{\cmark}\\

QueryAtomicFilterShadowing

& \multicolumn{1}{r}{0}
& \multicolumn{1}{r}{\xmark}

& \multicolumn{1}{r}{9.17}
& \multicolumn{1}{r}{}
& \multicolumn{1}{r}{188}
& \multicolumn{1}{r}{x\%}
& \multicolumn{1}{r}{\cmark}\\

QueryStreamedFilterShadowing

& \multicolumn{1}{r}{0}
& \multicolumn{1}{r}{\xmark}

& \multicolumn{1}{r}{}
& \multicolumn{1}{r}{}
& \multicolumn{1}{r}{}
& \multicolumn{1}{r}{x\%}
& \multicolumn{1}{r}{\cmark}\\

QueryStreamedLock

& \multicolumn{1}{r}{0}
& \multicolumn{1}{r}{\xmark}

& \multicolumn{1}{r}{}
& \multicolumn{1}{r}{}
& \multicolumn{1}{r}{}
& \multicolumn{1}{r}{x\%}
& \multicolumn{1}{r}{\cmark}\\

QueryStreamedBackUpNewStream

& \multicolumn{1}{r}{0}
& \multicolumn{1}{r}{\xmark}

& \multicolumn{1}{r}{}
& \multicolumn{1}{r}{}
& \multicolumn{1}{r}{}
& \multicolumn{1}{r}{x\%}
& \multicolumn{1}{r}{\cmark}\\

QueryStreamedSaveNewConfig

& \multicolumn{1}{r}{0}
& \multicolumn{1}{r}{\xmark}

& \multicolumn{1}{r}{}
& \multicolumn{1}{r}{}
& \multicolumn{1}{r}{}
& \multicolumn{1}{r}{x\%}
& \multicolumn{1}{r}{\cmark}\\

DeleteNoLeaveTombstonesEtag

& \multicolumn{1}{r}{0}
& \multicolumn{1}{r}{\xmark}

& \multicolumn{1}{r}{}
& \multicolumn{1}{r}{}
& \multicolumn{1}{r}{}
& \multicolumn{1}{r}{x\%}
& \multicolumn{1}{r}{\cmark}\\

DeletePrimaryKey

& \multicolumn{1}{r}{0}
& \multicolumn{1}{r}{\xmark}

& \multicolumn{1}{r}{}
& \multicolumn{1}{r}{}
& \multicolumn{1}{r}{}
& \multicolumn{1}{r}{x\%}
& \multicolumn{1}{r}{\cmark}\\

EnsurePartitionSwitchedFromPopulated

& \multicolumn{1}{r}{0}
& \multicolumn{1}{r}{\xmark}

& \multicolumn{1}{r}{}
& \multicolumn{1}{r}{}
& \multicolumn{1}{r}{}
& \multicolumn{1}{r}{x\%}
& \multicolumn{1}{r}{\cmark}\\

TombstoneOutputETag

& \multicolumn{1}{r}{0}
& \multicolumn{1}{r}{\xmark}

& \multicolumn{1}{r}{}
& \multicolumn{1}{r}{}
& \multicolumn{1}{r}{}
& \multicolumn{1}{r}{x\%}
& \multicolumn{1}{r}{\cmark}\\

MigrateSkipPreferOld

& \multicolumn{1}{r}{0}
& \multicolumn{1}{r}{\xmark}

& \multicolumn{1}{r}{}
& \multicolumn{1}{r}{}
& \multicolumn{1}{r}{}
& \multicolumn{1}{r}{x\%}
& \multicolumn{1}{r}{\cmark}\\

MigrateSkipUseNewWithTombstones

& \multicolumn{1}{r}{0}
& \multicolumn{1}{r}{\xmark}

& \multicolumn{1}{r}{}
& \multicolumn{1}{r}{}
& \multicolumn{1}{r}{}
& \multicolumn{1}{r}{x\%}
& \multicolumn{1}{r}{\cmark}\\

InsertBehindMigrator

& \multicolumn{1}{r}{0}
& \multicolumn{1}{r}{\xmark}

& \multicolumn{1}{r}{}
& \multicolumn{1}{r}{}
& \multicolumn{1}{r}{}
& \multicolumn{1}{r}{x\%}
& \multicolumn{1}{r}{\cmark}\\[0.1em]

\end{tabular}
\caption{Results from running the \psharp random and PCT systematic testing schedulers for 100,000 iterations. We report: time in seconds to find a bug (Time to Bug); number of scheduling points when a bug was found (\#SP); and if a bug was found with a particular scheduler (BF?).}
\label{tab:testing}
\end{table*}

Table~\ref{tab:testing} presents the results from running the \psharp systematic testing engine on each case study with an enabled bug using the random and the PCT schedulers. We configured the engine to perform 100,000 iterations. The random seed for both schedulers was generated in each iteration using the \texttt{DateTime.Now.Millisecond} API which represents the current time in milliseconds. The PCT scheduler was configured with a bug depth of 2 and a max number of scheduling steps to execute of 500. All reported times are in seconds.

How long you had to run the test to find the bug (if you can find it) and how long are the traces

How easy it is to pinpoint the error using the \psharp traces, comparing to traditional stress testing

Effort required to use the system

Controlled random scheduling has proven to be efficient for finding concurrency bugs~\cite{thomson2014sct, deligiannis2015psharp}.
